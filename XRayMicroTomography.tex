% Compile with:
% latexmk -pdf -pvc -interaction=nonstopmode
%\documentclass[aspectratio=169,10pt,draft]{beamer}
\documentclass[aspectratio=169,10pt]{beamer}
%\documentclass[handout,aspectratio=169,10pt]{beamer}
%\documentclass[notes=only,aspectratio=169,10pt]{beamer} % print out the notes
\usetheme{UniBern}
\title{X-ray microtomography}
\author{David Haberthür}
\institute{Institute of Anatomy\\University of Bern\\Switzerland}
\date{December 20, 2019 | \href{https://ilias.unibe.ch/ilias.php?ref_id=1555744&cmd=infoScreen&cmdClass=ilrepositorygui&cmdNode=y2&baseClass=ilrepositorygui}{9256-HS2019-0: Advanced Microscopy}}

%\includeonlyframes{current}
%then....
%\begin{frame}[label=current]
%\end{frame}

\usepackage{microtype}
\usepackage[backend=biber,
	style=numeric,
	url=false,
	isbn=true,
	maxbibnames=1,
	maxcitenames=1,
	sorting=none]{biblatex}
	\addbibresource{../../../Documents/library.bib}
\usepackage{graphicx}
\usepackage{tikz}
\usepackage{pgfplots}
	\pgfplotsset{compat=newest}
\usepackage[detect-all=true,
	range-phrase=--,
	range-units=single,
	binary-units=true,
	per-mode=symbol,
	per-symbol=/]{siunitx}
\usepackage[absolute,
	overlay]{textpos} %for the \source{} command
\usepackage{gitinfo2}
\usepackage[version=4]{mhchem}
\usepackage{xspace}
\usepackage{ccicons}
\usepackage{animate}
\usepackage{listings}
	\lstset{%
		frame=single,
		%backgroundcolor = \color{lightgray},
		basicstyle=\tiny\ttfamily
		}
\usepackage{multicol}
\usepackage{fontawesome5}

% change tikz font to slide font
% https://tex.stackexchange.com/a/33329/828
\usepackage[eulergreek]{sansmath}
	\pgfplotsset{tick label style = {font=\sansmath\sffamily},
		every axis label = {font=\sansmath\sffamily},
		legend style = {font=\sansmath\sffamily},
		label style = {font=\sansmath\sffamily}
		}

% Globally thicker lines in with tikz
% https://tex.stackexchange.com/a/206769/828
\tikzset{every picture/.style={semithick}}

% And thicker plots by default
% https://tex.stackexchange.com/a/235439/828
% https://tex.stackexchange.com/q/262486/828
\pgfplotsset{
	every axis plot/.append style={semithick},
	every axis/.append style={semithick}
	every axis plot post/.append style={every mark/.append style={semithick}}
}

% stripped-down plot styling
% Based on https://tex.stackexchange.com/a/155210/828
% And then start each plot with `\begin{axis}[tuftelike,'
\pgfkeys{%
	/pgfplots/simplified/.style={%
		tick style={major tick length=0pt},
%		separate axis lines,
%		axis x line*=bottom,
%		axis x line shift=10pt,
%		xlabel shift=1pt,
%		axis y line*=left,
%		axis y line shift=10pt,
		ymajorgrids=true,
		axis line style={draw=none},
%		ylabel shift=1pt
		}
	}

% Some often used abbreviations
\newcommand{\imsize}{0.618\linewidth} % set global image width
\newcommand{\everyframe}{1} % use only every nth frame for the movies
\newlength\imagewidth % needed for scalebars
\newlength\imagescale % needed for scalebars
\newcommand{\uct}{\si{\micro}CT\xspace} % make our life easier

% Acknowledge images just below them
% Based on https://tex.stackexchange.com/a/282637
\newcommand{\source}[2]{%
	\raisebox{-1.618ex}{\makebox[0pt][r]{\scriptsize\href{http://#1}{#1} #2}}
}

% Define us our custom footer
\defbeamertemplate{footline}{unibe}{%
	\hspace*{\fill}%
	v. \href{https://github.com/habi/20190605_BrukerUserMeeting/commit/\gitHash}{\gitAbbrevHash}\xspace|\xspace%
	p.\xspace\insertframenumber/\inserttotalframenumber%
	\hspace*{4ex}%
	\vskip2pt%
}
\setbeamertemplate{footline}[unibe]

% Format bibliography for beamer
% http://tex.stackexchange.com/a/10686/828
\renewbibmacro{in:}{}
% http://tex.stackexchange.com/a/13076/828
\AtEveryBibitem{%
	\clearfield{journaltitle}
	\clearfield{pages}
	\clearfield{volume}
	\clearfield{number}
	\clearname{editor}
	\clearfield{issn}
	\clearfield{year}
}
% No parentheses around the (now empty) year: https://tex.stackexchange.com/a/147537
\renewcommand{\bibopenparen}{\addcomma\addspace}
\renewcommand{\bibcloseparen}{\addcomma\addspace}

% Redefine \footcite based on https://tex.stackexchange.com/a/453528/828
\DeclareCiteCommand{\footcite}[\mkbibfootnote]
	{\usebibmacro{prenote}}
	  {\printnames[family-given]{labelname}%
	  \newunit
	  \printfield{doi}%
	  \newunit
	  \printlabeldateextra}
	  {\addsemicolon\space}
	  {\usebibmacro{postnote}}

% Show current section at begin of sections
\AtBeginSection[]{
	\begin{frame}{Outline}
	\small \tableofcontents[currentsection,currentsubsection, 
    hideothersubsections]
	\end{frame} 
}

% Slide transiton
%\addtobeamertemplate{background canvas}{\transfade[duration=0.5]}{}

% open in fullscreen
%\hypersetup{pdfpagemode=FullScreen}

% Move the text down a bit
% THIS IS A BIG HACK, IT SHOULD BE FIXED IN THE TEMPLATE
\addtobeamertemplate{frametitle}{}{\vspace*{1.5ex}}

% References as footnotes at the bottom of the slides
% https://tex.stackexchange.com/a/368760/828
\makeatletter
\renewcommand\@makefnmark{\xspace\hbox{\usebeamercolor[fg]{footnote mark}\usebeamerfont*{footnote mark}[\@thefnmark]}}
\renewcommand\@makefntext[1]{\tiny{\usebeamercolor[fg]{footnote mark}\usebeamerfont*{footnote mark}[\@thefnmark]}\enspace\usebeamerfont*{footnote} #1}
\makeatother

\begin{document}
% No footline on the title page
% http://tex.stackexchange.com/a/18829/828 helps us to achieve that
{%
	\setbeamertemplate{footline}{}%
	\begin{frame}%
		\maketitle
	\end{frame}%
}

% Alignment frame to test the ugly text-block-movement hack above
%\begin{frame}{Alignment frame}
%	\begin{tikzpicture}
%		\def\cut{3}
%		\draw [|-|,ultra thick] (0,\cut) -- (0,\textheight-\cut);
%		\draw [|-|,ultra thick] (0.5\textwidth,\cut) -- (0.5\textwidth,\textheight-\cut);
%		\draw [|-|,ultra thick] (\textwidth,\cut) -- (\textwidth,\textheight-\cut);
%	\end{tikzpicture}
%\end{frame}
\begin{frame}{Hello!}
	\begin{itemize}
		\item Office \href{http://osm.org/go/0CZwlGp3A?m}{B311} | \href{mailto:haberthuer@ana.unibe.ch?subject=Feedback\%20from\%20the\%20(micro)-tomography\%20lecture}{haberthuer@ana.unibe.ch}
		\item Master in Physics, then \href{https://boris.unibe.ch/2619/}{PhD in high resolution imaging of  the lung} at the Institute of Anatomy
		\item Post-Doc at the \href{https://www.psi.ch/sls/tomcat/}{TOMCAT beamline} of the \href{https://www.psi.ch/sls/}{Swiss Light Source} at the \href{https://www.psi.ch/}{Paul Scherrer Institute}
		\item Post-Doc at the \href{https://aan.unibe.ch}{Institute of Anatomy} in the \uct-group
		\begin{itemize}
			\item Ruslan Hlushchuk, David Haberthür, Oleksiy-Zakhar Khoma, Fluri Wieland, Carlos Correa Shokiche
		\end{itemize}			
		\item Biomedical research
		\begin{itemize}
			\item microangioCT~\footcite{Hlushchuk2018}: Tumor vasculature, angiogenesis in the heart, musculature and bones
			\item Cancer research: Melanoma
			\item Lung imaging: Tumor detection and classification
			\item Physiology: Zebrafish musculature and gills~\footcite{Messerli2019}
			\item SkyScan 1172 \& 1272
		\end{itemize}
	\end{itemize}		
\end{frame}

\begin{frame}{Contents}
%    \begin{multicols}{2}
	\tableofcontents
%    \end{multicols}
\end{frame}

\section{Biomedical imaging}
\begin{frame}[allowframebreaks]{Biomedical imaging}
	\centering
	\includegraphics[height=0.618\textheight]{./images/Sagittal_brain_MRI}
	\source{w.wiki/7g4}{\ccbysa}
	% ADD IMAGE OF MOUSE SKULL	
	\framebreak
	\begin{itemize}
		\item Non-destructive insight
		\item Biological samples
	    \end{itemize}
\end{frame}

\section{Imaging}

\begin{frame}{Wavelength \& Scale}
    \begin{figure}
	\centering
	\includegraphics[height=0.618\textheight]{./images/{{2000px-Electromagnetic_spectrum_with_sources.svg}}}
	\source{w.wiki/7fz}{\ccbysa}
    \end{figure}
\end{frame}
	
\begin{frame}{Imaging methods}
	\begin{itemize}
		\item Light microscopy
		\item X-ray imaging
		\item Electron microscopy
	\end{itemize}
\end{frame}

\section{Tomography}
\begin{frame}{Tomography}
	\centering
	\includegraphics[height=0.618\textheight]{./images/Ct-internals.jpg}
	\source{w.wiki/7g6}{\ccbysa}
\end{frame}

\begin{frame}{Scanner}
	\url{https://www.youtube.com/watch?v=2CWpZKuy-NE}
\end{frame}

\begin{frame}{What is going on here?}
	\centering
	\includegraphics[height=0.618\textheight]{./images/3D_Computed_Tomography}
	\source{w.wiki/7g3}{\ccbysa}
\end{frame}

\subsection{X-ray production}
\begin{frame}{X-ray generation}
	\begin{itemize}
		\item How are x-rays generated
		\item  Why do we need them
	\end{itemize}
\end{frame}

\subsection{Interaction of x-rays with matter}

\subsection{History}
\cite{Cormack1963a}
% FIRST, SECOND AND THIRD GENERATION CT SCANNERS

% HOW DO WE GET FROM XRAY TO DATA?


\subsection{A scan, from start to finish}
\begin{frame}[allowframebreaks]{Projections}
	sample preparation
	study design
\end{frame}

\begin{frame}[allowframebreaks]{Projections}
	% Overview
	movies/KP-TNIKWT02\_projections.mp4
	\framebreak
	% What is going on?
	\begin{itemize}
		\item What happens after penetration of the sample?
		\item Attenuation
		\item Conversion to visible light by Scintillator
		\item Detection, recording
	\end{itemize}
\end{frame}

\begin{frame}[allowframebreaks]{Reconstructions}
	%Overview
	movies/KP-TNIKWT02\_reconstructions.mp4
	\framebreak
	\begin{itemize}
		\item Radon Transformation
		\item Filtered back projection
		\item Fan beam reconstruction
		\item Corrections (beam hardening, etc.)
		\item Writing to stack
	\end{itemize}
\end{frame}

\renewcommand{\imsize}{1.6\columnwidth}
\begin{frame}{Machinery}
	\begin{columns}
		\begin{column}{0.5\linewidth}
			\begin{itemize}
				\item<1-> Hospital CT
				\item<2-> Lab CT
				\item<3-> Desktop CT
				\item<4-> Synchrotron CT
			\end{itemize}
		\end{column}
		\begin{column}{0.5\linewidth}
			\includegraphics<1>[height=0.618\textheight]{./images/24324062640_751e011e1a_o}%
			\only<1>{\source{flic.kr/p/D4rbom}{\ccbyncsa}}%
			\includegraphics<2>[height=0.618\textheight]{./images/9459311320_516179207a_o}
			\only<2>{\source{flic.kr/p/fpTrGu}{\ccbysa}}%
			\includegraphics<3>[width=\imsize]{./images/1272}%
			\only<3>{\source{bruker.com/skyscan1272}{}}%
			\includegraphics<4>[width=\imsize]{./images/4563733710_f632792416_b}%
			\only<4>{\source{https://flic.kr/p/7Xhk2Y}{\ccbync}}%
		\end{column}
	\end{columns}
\end{frame}

\note{From \url{https://www.bruker.com/products/microtomography/micro-ct-for-sample-scanning/x-ray-micro-ct-microtomography.html}: 
Micro computed tomography or "micro-CT" is x-ray imaging in 3D, by the same method used in hospital CT (or "CAT") scans, but on a small scale with massively increased resolution. It really represents 3D microscopy, where very fine scale internal structure of objects is imaged non-destructively. No sample preparation, no staining, no thin slicing - a single scan will image your sample's complete internal 3D structure at high resolution, plus you get your intact sample back at the end!

How does micro-CT work? A micro-focus x-ray source illuminates the object and a planar x-ray detector collects magnified projection images. Based on hundreds of angular views acquired while the object rotates, a computer synthesizes a stack of virtual cross section slices through the object. You can then scroll through the cross sections, interpolating sections along different planes, to inspect the internal structure. Selecting simple or complex volumes of interest, you can measure 3D morphometric parameters and create realistic visual models for virtual travel within the object.
Methods1

Bruker microCT can genuinely claim to be at the fore-front of the development of high performance micro-CT technology. Our research and development of 3D x-ray microscopy started in the early 1980s. This led to the first micro-CT imaging results being obtained in 1983-1987 and published in scientific journals and international conferences proceedings. Building on this early work, Bruker-microCT was founded in 1996, and within a year we were manufacturing a commercially available micro-CT scanner with spatial resolution in the micron range. In 2001 we produced the first high-resolution in vivo micro-CT scanner for small animal imaging. And in 2005 Bruker-microCTbecame the world's only supplier of a laboratory nano-CT scanner with submicron spatial resolution. Responding to demand from the growing community of micro-CT users, we are continually active in research and development into new methods for non-destructive 3D microscopy.}

% FAN BEAM -> BRUKER
% PARALLEL BEAM -> TOMCAT
% HELICAL/SPIRAL CT
% MULTISLICE CT -> HOSPITAL
% SCREENING/IN VIVO/DENSITOMETRY/ETC.

\subsection{Imaging performance}
% RESOLUTION/PIXEL SIZE -> PHANTOM?
% POINT SPREAD FUNCTION
% NOISE
% TEMPORAL RESOLUTION (ONLY FOR HUMAN MEDICAL CT)


\section{Image processing}
\subsection{Image display}
% MIP
% SURFACE RENDERING

\begin{frame}{What to use?}
	\begin{itemize}
		\item ImageJ
		\item \faPython
		\item See lecture of Guillaume Witz
	\end{itemize}
\end{frame}

\begin{frame}{Big data}
	\begin{itemize}
		\item TOMCAT 2560\(\times\)2160 @ 1500 fps \textgreater{} 8 GB/s
		\item Desktop uCT \textgreater{} 100s of GBs in a day
	\end{itemize}
\end{frame}

\begin{frame}{Reproducible research}
	\begin{itemize}
		\item Git
		\item Jupyter
		\item Data repositories \(\rightarrow\) Sharing is caring!
	\end{itemize}
\end{frame}

\begin{frame}{\href{https://en.wikipedia.org/wiki/Colophon_(publishing)}{Colophon}}
	\begin{itemize} 
		\item \textsc{beamer} presentation crafted in \LaTeX with the (slightly adapted) \href{http://intern.unibe.ch/dienstleistungen/corporate_design_und_vorlagen/praesentationen/index_ger.html}{template from from \emph{Corporate Design und Vorlagen} of the University of Bern}.
		\begin{itemize}
			\item Full source available on \href{https://github.com/habi/lecture_microtomography/}{GitHub}: git.io/fjpP7
			\item Spotted an error? Please file an \href{https://github.com/habi/lecture_microtomography/issues}{issue} (git.io/fjpPb) or (even better) submit a \href{https://github.com/habi/lecture_microtomography/pulls}{pull request} (git.io/fjpPN).
		\end{itemize}
	\end{itemize}
\end{frame}

\begin{frame}{References}
	\renewcommand*{\bibfont}{\scriptsize}
	\setbeamertemplate{bibliography item}{\insertbiblabel}
	\printbibliography
\end{frame}

\end{document}
% Compile with:
% latexmk -pdf -pvc -interaction=nonstopmode
%\documentclass[aspectratio=169,10pt,draft]{beamer}
%\documentclass[aspectratio=169,10pt]{beamer}
\documentclass[handout,aspectratio=169,10pt]{beamer}
%\documentclass[notes=only,aspectratio=169,10pt]{beamer} % print out the notes
\usetheme{UniBern}
\title{X-ray microtomography}
\author{David Haberthür}
\institute{Institute of Anatomy\\University of Bern\\Switzerland}
\date{December 20, 2019 | \href{https://ilias.unibe.ch/ilias.php?ref_id=1555744&cmd=infoScreen&cmdClass=ilrepositorygui&cmdNode=y2&baseClass=ilrepositorygui}{9256-HS2019-0: Advanced Microscopy}}

%\includeonlyframes{current}
%then....
%\begin{frame}[label=current]
%\end{frame}

\usepackage{microtype}
\usepackage[backend=biber,
	style=numeric,
	url=false,
	isbn=true,
	maxbibnames=1,
	maxcitenames=1,
	sorting=none]{biblatex}
	\addbibresource{../../../Documents/library.bib}
\usepackage{graphicx}
\usepackage{tikz}
\usepackage{pgfplots}
	\pgfplotsset{compat=newest}
\usepackage[detect-all=true,
	range-phrase=--,
	range-units=single,
	binary-units=true,
	per-mode=symbol,
	per-symbol=/]{siunitx}
\usepackage[absolute,
	overlay]{textpos} %for the \source{} command
\usepackage{gitinfo2}
\usepackage[version=4]{mhchem}
\usepackage{xspace}
\usepackage{ccicons}
\usepackage{animate}
\usepackage{listings}
	\lstset{%
		frame=single,
		%backgroundcolor = \color{lightgray},
		basicstyle=\tiny\ttfamily
		}
\usepackage{multicol}
\usepackage{fontawesome5}
\usepackage{csquotes}

% change tikz font to slide font
% https://tex.stackexchange.com/a/33329/828
\usepackage[eulergreek]{sansmath}
	\pgfplotsset{tick label style = {font=\sansmath\sffamily},
		every axis label = {font=\sansmath\sffamily},
		legend style = {font=\sansmath\sffamily},
		label style = {font=\sansmath\sffamily}
		}

% Globally thicker lines in with tikz
% https://tex.stackexchange.com/a/206769/828
\tikzset{every picture/.style={semithick}}

% And thicker plots by default
% https://tex.stackexchange.com/a/235439/828
% https://tex.stackexchange.com/q/262486/828
\pgfplotsset{
	every axis plot/.append style={semithick},
	every axis/.append style={semithick}
	every axis plot post/.append style={every mark/.append style={semithick}}
}

% stripped-down plot styling
% Based on https://tex.stackexchange.com/a/155210/828
% And then start each plot with `\begin{axis}[tuftelike,'
\pgfkeys{%
	/pgfplots/simplified/.style={%
		tick style={major tick length=0pt},
%		separate axis lines,
%		axis x line*=bottom,
%		axis x line shift=10pt,
%		xlabel shift=1pt,
%		axis y line*=left,
%		axis y line shift=10pt,
		ymajorgrids=true,
		axis line style={draw=none},
%		ylabel shift=1pt
		}
	}

% Some often used abbreviations
\newcommand{\imsize}{0.618\linewidth} % set global image width
\newcommand{\everyframe}{1} % use only every nth frame for the movies
\newlength\imagewidth % needed for scalebars
\newlength\imagescale % needed for scalebars
\newcommand{\uct}{\si{\micro}CT\xspace} % make our life easier

% Acknowledge images just below them
% Based on https://tex.stackexchange.com/a/282637
\newcommand{\source}[2]{%
	\raisebox{-1.618ex}{\makebox[0pt][r]{\scriptsize\href{http://#1}{#1} #2}}
}

% Define us our custom footer
\defbeamertemplate{footline}{unibe}{%
	\hspace*{\fill}%
	v. \href{https://github.com/habi/20190605_BrukerUserMeeting/commit/\gitHash}{\gitAbbrevHash}\xspace|\xspace%
	p.\xspace\insertframenumber/\inserttotalframenumber%
	\hspace*{4ex}%
	\vskip2pt%
}
\setbeamertemplate{footline}[unibe]

% Format bibliography for beamer
% http://tex.stackexchange.com/a/10686/828
\renewbibmacro{in:}{}
% http://tex.stackexchange.com/a/13076/828
\AtEveryBibitem{%
	\clearfield{journaltitle}
	\clearfield{pages}
	\clearfield{volume}
	\clearfield{number}
	\clearname{editor}
	\clearfield{issn}
	\clearfield{year}
}
% No parentheses around the (now empty) year: https://tex.stackexchange.com/a/147537
\renewcommand{\bibopenparen}{\addcomma\addspace}
\renewcommand{\bibcloseparen}{\addcomma\addspace}

% Redefine \footcite based on https://tex.stackexchange.com/a/453528/828
\DeclareCiteCommand{\footcite}[\mkbibfootnote]
	{\usebibmacro{prenote}}
	  {\printnames[family-given]{labelname}%
	  \newunit
	  \printfield{doi}%
	  \newunit
	  \printlabeldateextra}
	  {\addsemicolon\space}
	  {\usebibmacro{postnote}}

% Show current section at begin of sections
%\AtBeginSection[]{
%	\begin{frame}{Outline}
%	\small \tableofcontents[currentsection, hideothersubsections]
%	\end{frame} 
%}

% Slide transiton
%\addtobeamertemplate{background canvas}{\transfade[duration=0.5]}{}

% open in fullscreen
%\hypersetup{pdfpagemode=FullScreen}

% Move the text down a bit
% THIS IS A BIG HACK, IT SHOULD BE FIXED IN THE TEMPLATE
\addtobeamertemplate{frametitle}{}{\vspace*{1.5ex}}

% References as footnotes at the bottom of the slides
% https://tex.stackexchange.com/a/368760/828
\makeatletter
\renewcommand\@makefnmark{\hbox{\usebeamercolor[fg]{footnote mark}\usebeamerfont*{footnote mark}[\@thefnmark]}}
\renewcommand\@makefntext[1]{\tiny{\usebeamercolor[fg]{footnote mark}\usebeamerfont*{footnote mark}[\@thefnmark]}\enspace\usebeamerfont*{footnote} #1}
\makeatother

\begin{document}
% No footline on the title page
% http://tex.stackexchange.com/a/18829/828 helps us to achieve that
{%
	\setbeamertemplate{footline}{}%
	\begin{frame}%
		\maketitle
	\end{frame}%
}

% Alignment frame to test the ugly text-block-movement hack above
%\begin{frame}
%	\frametitle{Alignment frame}
%	\begin{tikzpicture}
%		\def\cut{3}
%		\draw [|-|,ultra thick] (0,\cut) -- (0,\textheight-\cut);
%		\draw [|-|,ultra thick] (0.5\textwidth,\cut) -- (0.5\textwidth,\textheight-\cut);
%		\draw [|-|,ultra thick] (\textwidth,\cut) -- (\textwidth,\textheight-\cut);
%	\end{tikzpicture}
%\end{frame}

\begin{frame}
	\frametitle{Hello!}
	\begin{itemize}
		\item<1-> University of Bern, Switzerland
		\item<1-> Institute of Anatomy
		\item<1-> \uct-group: Ruslan Hlushchuk, David Haberthür, Oleksiy-Zakhar Khoma, Fluri Wieland, Carlos Correa Shokiche
		\item<1-> Biomedical research
		\begin{itemize}
			\item microangioCT~\footcite{Hlushchuk2018}: Tumor vasculature, angiogenesis in the heart, musculature and bones
			\item Cancer research: Melanoma
			\item Lung imaging: Tumor detection and classification
			\item Physiology: Zebrafish musculature and gills
		\end{itemize}
		\item<1-> SkyScan 1172 \& 1272
	\end{itemize}
\end{frame}

\begin{frame}
	\frametitle{Hello!}
\begin{itemize}
\item
  \href{mailto:haberthuer@ana.unibe.ch?subject=Feedback\%20from\%20the\%20(micro)-tomography\%20lecture}{haberthuer@ana.unibe.ch}
\item
  Office \href{http://osm.org/go/0CZwlGp3A?m}{B311}
\item
  Master in Physics
\item
  \href{https://boris.unibe.ch/2619/}{PhD in high resolution imaging of
  the lung}
\item
  Postdoctoral researcher at the
  \href{https://www.psi.ch/sls/tomcat/}{TOMCAT beamline} of the
  \href{https://www.psi.ch/sls/}{Swiss Light Source} at the
  \href{https://www.psi.ch/}{Paul Scherrer Institute}
\item
  Tomography is what I do day-in, day-out
\end{itemize}
\end{frame}

\begin{frame}{Contents}
    \begin{multicols}{2}
	\tableofcontents
    \end{multicols}
\end{frame}

\section{Biomedical imaging}
\begin{frame}
	\frametitle{Biomedical imaging}
	\centering
	\includegraphics[height=0.618\textheight]{./images/Sagittal_brain_MRI}
	\source{w.wiki/7g4}{\ccbysa}
\end{frame}

\begin{frame}
    \frametitle{Why}
    \begin{itemize}
	\item Non-destructive insight
	\item Biological samples
    \end{itemize}
\end{frame}

\begin{frame}
    \frametitle{Wavelength \& Scale}
    \begin{figure}
	\centering
	\includegraphics[height=0.618\textheight]{/home/habi/P/Talks/Lectures/MicroTomography/images/{{2000px-Electromagnetic_spectrum_with_sources.svg}}}
	\source{w.wiki/7fz}{\ccbysa}
    \end{figure}
\end{frame}
	
\begin{frame}
	\frametitle{Imaging methods}
	\begin{itemize}
		\item Light microscopy
		\item X-ray imaging
		\item Electron microscopy
	\end{itemize}
\end{frame}

\section{Tomography}
\begin{frame}
	\frametitle{Tomography}
	\centering
	\includegraphics[height=0.618\textheight]{/home/habi/P/Talks/Lectures/MicroTomography/images/Ct-internals.jpg}
	\source{w.wiki/7g6}{\ccbysa}
\end{frame}

\begin{frame}
	\frametitle{Scanner}
	\url{https://www.youtube.com/watch?v=2CWpZKuy-NE}
\end{frame}

\begin{frame}
	\frametitle{What is going on here?}
	\centering
	\includegraphics[height=0.618\textheight]{./images/3D_Computed_Tomography}
	\source{w.wiki/7g3}{\ccbysa}
\end{frame}

\begin{frame}
	\frametitle{X-ray generation}
	\begin{itemize}
		\item How are x-rays generated
		\item  Why do we need them
	\end{itemize}
\end{frame}

\begin{frame}
\hypertarget{projections}{\subsection{Projections}\label{projections}}
%movies/KP-TNIKWT02_projections.mp4
\end{frame}

\begin{frame}
\hypertarget{projections-how}{\subsection{Projections, how?}\label{projections-how}}

\begin{itemize}
	\item What happens after penetration of the sample?
	\item Attenuation
	\item Conversion to visible light by Scintillator
	\item Detection, recording
\end{itemize}

\end{frame}

\begin{frame}
\hypertarget{reconstructions}{\subsection{Reconstructions}\label{reconstructions}}
%movies/KP-TNIKWT02_reconstructions.mp4
\end{frame}

\begin{frame}
\hypertarget{reconstruction-how}{\subsection{Reconstruction, how?}\label{reconstruction-how}}

\begin{itemize}
	\item Radon Transformation
	\item Filtered back projection
	\item Corrections
	\item Writing to stack
\end{itemize}
\end{frame}


\begin{frame}
\hypertarget{machinery}{\subsection{Machinery}\label{machinery}}

\begin{itemize}
	\item Hospital CT
	\item Lab CT
	\item Synchrotron
	\item Desktop microCT
\end{itemize}

	\includegraphics{./images/9459311320_516179207a_o}
	\source{flic.kr/p/fpTrGu}{\ccbysa}
\end{frame}

\section{Image processing}

\subsection{What to use?}

\begin{frame}
	\begin{itemize}
		\item ImageJ
		\item Python
	\end{itemize}
\end{frame}

\hypertarget{big-data}{\subsection{Big data}\label{big-data}}

\begin{frame}

\begin{itemize}
	\item TOMCAT 2560\(\times\)2160 @ 1500 fps \textgreater{} 8 GB/s
	\item Desktop uCT \textgreater{} 100s of GBs in a day
\end{itemize}
\end{frame}

\hypertarget{image-processing-1}{\subsection{Image processing}\label{image-processing-1}}

\begin{frame}

\begin{itemize}
	\item See lecture of Guillaume Witz
\end{itemize}
\end{frame}

\hypertarget{reproducible-research}{\subsection{Reproducible research}\label{reproducible-research}}
\begin{frame}

\begin{itemize}
	\item  Reproducibility?
\end{itemize}
\end{frame}


\hypertarget{thank-you-questions}{\section{Thank you! Questions?}\label{thank-you-questions}}

\hypertarget{colophon}{\subsection{\texorpdfstring{\href{https://en.wikipedia.org/wiki/Colophon_(publishing)}{Colophon}}{Colophon}}\label{colophon}}

\begin{frame}

\begin{itemize}
	\item Written in \href{https://daringfireball.net/projects/markdown/}{Markdown}.
	\item Versioned with \href{https://git-scm.com/}{git}, public copy hosted by
  \href{https://github.com/habi/lecture_microtomography/}{GitHub}.
	\item Converted to a \href{https://revealjs.com/}{reveal.js} presentation with \href{https://pandoc.org/}{pandoc} (see code snippet in \href{https://github.com/habi/lecture_microtomography/blob/master/README.md}{README.md}).
	\item \href{http://habi.github.io/lecture_microtomography}{Public presentation} hosted by \href{https://pages.github.com/}{GitHub Pages}.
	\item Spotted an error? Please file an \href{https://github.com/habi/lecture_microtomography/issues}{issue} or (even better) submit a \href{https://github.com/habi/lecture_microtomography/pulls}{pull request}.
\end{itemize}
\end{frame}

\begin{frame}
	\frametitle{References}
	\renewcommand*{\bibfont}{\scriptsize}
	\setbeamertemplate{bibliography item}{\insertbiblabel}
	\printbibliography
\end{frame}

\begin{frame}
	\frametitle{Colophon}
	\begin{itemize}
		\item \href{https://github.com/habi/20190605_BrukerUserMeeting}{The presentation (\LaTeX/\textsc{beamer}) is available on GitHub}: git.io/fjuo5
		\item \href{http://intern.unibe.ch/dienstleistungen/corporate_design_und_vorlagen/praesentationen/index_ger.html}{The presentation template is (slightly adapted) from \emph{Corporate Design und Vorlagen, University of Bern}}
	\end{itemize}
\end{frame}

\end{document}

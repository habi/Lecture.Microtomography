% Compile with:
% latexmk -pdf -pvc -interaction=nonstopmode
%\documentclass[aspectratio=169,10pt,draft]{beamer}
\documentclass[aspectratio=169,10pt]{beamer}
%\documentclass[handout,aspectratio=169,10pt]{beamer}
%\documentclass[notes=only,aspectratio=169,10pt]{beamer} % print out the notes

\usetheme{UniBern}

\title{X-ray microtomography}
\author{David Haberthür}
\date{December 20, 2019 | \href{https://ilias.unibe.ch/ilias.php?ref_id=1555744&cmd=infoScreen&cmdClass=ilrepositorygui&cmdNode=y2&baseClass=ilrepositorygui}{9256-HS2019-0: Advanced Microscopy}}

%\includeonlyframes{current}
%then....
%\begin{frame}[label=current]
%\end{frame}

\usepackage{microtype}
\usepackage[backend=biber,
	style=numeric,
	url=false,
	isbn=true,
	maxbibnames=1,
	maxcitenames=1,
	sorting=none]{biblatex}
	\addbibresource{../../../Documents/library.bib}
\usepackage{graphicx}
\usepackage{tikz}
\usepackage{pgfplots}
	\pgfplotsset{compat=newest}
\usepackage[detect-all=true,
	range-phrase=--,
	range-units=single,
	binary-units=true,
	per-mode=symbol,
	per-symbol=/]{siunitx}
\usepackage[absolute,
	overlay]{textpos} %for the \source{} command
\usepackage{gitinfo2}
\usepackage[version=4]{mhchem}
\usepackage{xspace}
\usepackage{ccicons}
\usepackage{animate}
\usepackage{lipsum}
\usepackage{listings}
	\lstset{%
		frame=single,
		%backgroundcolor = \color{lightgray},
		basicstyle=\tiny\ttfamily
		}
\usepackage{multicol}
\usepackage{fontawesome5}

% change tikz font to slide font
% https://tex.stackexchange.com/a/33329/828
\usepackage[eulergreek]{sansmath}
	\pgfplotsset{tick label style = {font=\sansmath\sffamily},
		every axis label = {font=\sansmath\sffamily},
		legend style = {font=\sansmath\sffamily},
		label style = {font=\sansmath\sffamily}
		}

% Globally thicker lines in with tikz
% https://tex.stackexchange.com/a/206769/828
\tikzset{every picture/.style={semithick}}

% And thicker plots by default
% https://tex.stackexchange.com/a/235439/828
% https://tex.stackexchange.com/q/262486/828
\pgfplotsset{
	every axis plot/.append style={semithick},
	every axis/.append style={semithick}
	every axis plot post/.append style={every mark/.append style={semithick}}
}

% stripped-down plot styling
% Based on https://tex.stackexchange.com/a/155210/828
% And then start each plot with `\begin{axis}[tuftelike,'
\pgfkeys{%
	/pgfplots/simplified/.style={%
		tick style={major tick length=0pt},
%		separate axis lines,
%		axis x line*=bottom,
%		axis x line shift=10pt,
%		xlabel shift=1pt,
%		axis y line*=left,
%		axis y line shift=10pt,
		ymajorgrids=true,
		axis line style={draw=none},
%		ylabel shift=1pt
		}
	}

% Some often used abbreviations
\newcommand{\imwidth}{0.618\linewidth}% set global image width
\newcommand{\imheight}{0.7\paperheight}% set global image height
\newcommand{\everyframe}{5}% use only every nth frame for the movies
\newlength\imagewidth% needed for scalebars
\newlength\imagescale% needed for scalebars
\newcommand{\eg}{e.\,g.\xspace}
\newcommand{\ie}{i.\,e.\xspace}
\newcommand{\uct}{\si{\micro}CT\xspace}% make our life easier

% Acknowledge images just below them
% Based on https://tex.stackexchange.com/a/282637
\newcommand{\source}[2]{\raisebox{-1.618ex}{\makebox[0pt][r]{\scriptsize\href{http://#1}{#1} #2}}}

%% Define us our custom footer
\defbeamertemplate{footline}{unibe}{%
	\hspace*{\fill}%
	v. \href{https://github.com/habi/lecture.microtomography/commit/\gitHash}{\gitAbbrevHash}\xspace|\xspace%
	p.\xspace\insertframenumber/\inserttotalframenumber%
	\hspace*{4ex}%
	\vskip2pt%
}
\setbeamertemplate{footline}[unibe]

% Format bibliography for beamer
% http://tex.stackexchange.com/a/10686/828
\renewbibmacro{in:}{}
% http://tex.stackexchange.com/a/13076/828
\AtEveryBibitem{%
	\clearfield{journaltitle}
	\clearfield{pages}
	\clearfield{volume}
	\clearfield{number}
	\clearname{editor}
	\clearfield{issn}
	\clearfield{year}
}
% No parentheses around the (now empty) year: https://tex.stackexchange.com/a/147537
\renewcommand{\bibopenparen}{\addcomma\addspace}
\renewcommand{\bibcloseparen}{\addcomma\addspace}

% Redefine \footcite based on https://tex.stackexchange.com/a/453528/828
\DeclareCiteCommand{\footcite}[\mkbibfootnote]{%
	\usebibmacro{prenote}}{%
		\printnames[family-given]{labelname}%
		\newunit%
		\printfield{doi}%
		\newunit%
		\printlabeldateextra%
	}{\addsemicolon\space}{%
		\usebibmacro{postnote}%
	}%

% Show current section at begin of sections
\AtBeginSection[]{
	\begin{frame}{Outline}
		\small \tableofcontents[currentsection,currentsubsection,hideothersubsections]
	\end{frame}
}

% Slide transiton
%\addtobeamertemplate{background canvas}{\transfade[duration=0.5]}{}

% open in fullscreen
%\hypersetup{pdfpagemode=FullScreen}

% Move the text down a bit
% THIS IS A BIG HACK, IT SHOULD BE FIXED IN THE TEMPLATE
\addtobeamertemplate{frametitle}{}{\vspace*{0.75ex}}

% References as footnotes at the bottom of the slides
% https://tex.stackexchange.com/a/368760/828
\makeatletter
\renewcommand\@makefnmark{\xspace\hbox{\usebeamercolor[fg]{footnote mark}\usebeamerfont*{footnote mark}[\@thefnmark]}}
\renewcommand\@makefntext[1]{\tiny{\usebeamercolor[fg]{footnote mark}\usebeamerfont*{footnote mark}[\@thefnmark]}\enspace\usebeamerfont*{footnote} #1}
\makeatother

\begin{document}
% No footline on the title page
% http://tex.stackexchange.com/a/18829/828 helps us to achieve that
{%
	\setbeamertemplate{footline}{}%
	\begin{frame}%
		\maketitle
	\end{frame}%
}

% Alignment frames to test the ugly text-block-movement hack above
%\begin{frame}[label=current]{Alignment frame}
%	\begin{tikzpicture}%
%		\def\cut{3}%
%		\draw [|-|,ultra thick] (0,\cut) -- (0,\textheight-\cut);%
%		\draw [<->,ultra thick] (0.5\textwidth,\cut) -- (0.5\textwidth,\textheight-\cut);%
%		\draw [|-|,ultra thick] (\textwidth,\cut) -- (\textwidth,\textheight-\cut);%
%	\end{tikzpicture}%
%\end{frame}
%
%\begin{frame}[allowframebreaks,label=current]{Alignment frame II}
%	\lipsum[1-50]
%\end{frame}

\begin{frame}{Hello!}
	\begin{itemize}
		\item Office \href{http://osm.org/go/0CZwlGp3A?m}{B311} | \href{mailto:haberthuer@ana.unibe.ch?subject=Feedback\%20from\%20the\%20(micro)-tomography\%20lecture}{haberthuer@ana.unibe.ch}
		\item Master in Physics, then \href{https://boris.unibe.ch/2619/}{PhD in high resolution imaging of the lung} at the Institute of Anatomy
		\item Post-Doc at the \href{https://www.psi.ch/sls/tomcat/}{TOMCAT beamline} of the \href{https://www.psi.ch/sls/}{Swiss Light Source} at the \href{https://www.psi.ch/}{Paul Scherrer Institute}
		\item Post-Doc at the \href{https://ana.unibe.ch}{Institute of Anatomy} in the \uct-group
		\begin{itemize}
			\item Ruslan Hlushchuk, David Haberthür, Oleksiy-Zakhar Khoma, Fluri Wieland, Carlos Correa Shokiche
		\end{itemize}			
		\item Biomedical research
		\begin{itemize}
			\item microangioCT~\footcite{Hlushchuk2018}: Tumor vasculature, angiogenesis in the heart, musculature and bones
			\item Cancer research: Melanoma
			\item Lung imaging: Tumor detection and classification
			\item Physiology: Zebrafish musculature and gills~\footcite{Messerli2019}
			\item SkyScan 1172 \& 1272
		\end{itemize}
	\end{itemize}		
\end{frame}

\begin{frame}{Contents}
%	\begin{multicols}{2}
	\tableofcontents
%	\end{multicols}
\end{frame}

\section{Biomedical imaging}
\begin{frame}{Biomedical imaging}
	\centering
	\only<1>{%
		\includegraphics[height=0.618\textheight]{./images/Sagittal_brain_MRI}
		\source{w.wiki/7g4}{\ccbysa}
		}
	\only<2>{%
		\begin{tikzpicture}[remember picture,overlay]%
			\node at (current page.center){%
				\animategraphics[palindrome,autoplay,width=\paperwidth,every=\everyframe]{25}{./movies/mouse_skull/mouse_skull}{000}{470}%
				};%
		\end{tikzpicture}%
		}
\end{frame}

\begin{frame}{Biomedical imaging}
	\begin{itemize}
		\item (Small) Biological samples
		\item Non-destructive insights into the samples
		\item Medical research
	\end{itemize}
\end{frame}

\section{Imaging}
\begin{frame}{Wavelength \& Scale}
	\centering
	\begin{figure}
		\includegraphics[height=0.618\textheight]{./images/{{2000px-Electromagnetic_spectrum_with_sources.svg}}}
		\source{w.wiki/7fz}{\ccbysa}
	\end{figure}
\end{frame}
	
\begin{frame}{Imaging methods}
	\begin{itemize}
		\item Light microscopy: see \href{https://ilias.unibe.ch/goto_ilias3_unibe_sess_1555739.html}{lecture of Nadia Mercader Huber}
		\item X-ray imaging
		\item Electron microscopy: see lectures on
			\href{https://ilias.unibe.ch/goto_ilias3_unibe_sess_1555738.html}{Transmission Electron Microscopy by Dimitri Vanhecke},
			\href{https://ilias.unibe.ch/goto_ilias3_unibe_sess_1555742.html}{Scanning Electron Microscopy by Michael Stoffel} and
			\href{https://ilias.unibe.ch/goto_ilias3_unibe_sess_1555743.html}{Cryoelectron Microscopy \& Serial Block Face SEM by Ioan Iacovache}.
	\end{itemize}
\end{frame}

\subsection{Tomography}
\begin{frame}{CT-Scanner}
	\centering
	\animategraphics[palindrome,autoplay,height=0.618\textheight,every=\everyframe]{25}{./movies/ct-scanner/ct-scanner}{1200}{1480}
	\source{youtu.be/2CWpZKuy-NE}{}
	\note{From \url{https://www.bruker.com/products/microtomography/micro-ct-for-sample-scanning/x-ray-micro-ct-microtomography.html}: 
		Micro computed tomography or micro-CT is x-ray imaging in 3D, by the same method used in hospital CT (or CAT) scans, but on a small scale with massively increased resolution.
		It really represents 3D microscopy, where very fine scale internal structure of objects is imaged non-destructively.
		No sample preparation, no staining, no thin slicing - a single scan will image your sample's complete internal 3D structure at high resolution, plus you get your intact sample back at the end!}
\end{frame}

\begin{frame}{What is happening?}
	\centering
	\includegraphics[height=0.618\textheight]{./images/3D_Computed_Tomography}
	\source{w.wiki/7g3}{\ccbysa}
\end{frame}

\subsection{X-ray production}
\begin{frame}{X-ray generation}
	\begin{itemize}
		\item How are x-rays generated
		\item Why do we need them
	\end{itemize}
\end{frame}

\subsection{Interaction of x-rays with matter}

\subsection{History}
\begin{frame}{History}
	\begin{columns}
		\begin{column}{0.5\linewidth}
			\begin{itemize}
				\item<1-> Some history is found in~\cite{Cormack1963a}
				\item<2-> First, second and third generation of scanners
				\item<5> HOW DO WE GET FROM XRAY TO DATA?
			\end{itemize}
		\end{column}
		\begin{column}{0.5\linewidth}
			\begin{itemize}
				\item<2> First generation image
				\item<3> Second generation image
				\item<4> Third generation image
			\end{itemize}
		\end{column}
	\end{columns}		
\end{frame}

\subsection{A scan, from start to finish}
\begin{frame}{Projections}
	\begin{itemize}
		\item sample preparation
		\item study design
	\end{itemize}
\end{frame}

\begin{frame}[label=current]{Projections}
	\centering
	\animategraphics[autoplay,loop,height=\imheight,every=\everyframe]{25}{./movies/scan/projections/KP-TNIKWT02_240_projections_of_940_800_px_}{000}{313}
\end{frame}

\begin{frame}[label=asdfcurrent]{Projections}
	\begin{itemize}
		\item A (micro-focus) x-ray source illuminates the object
		\item A planar x-ray detector collects magnified projection images.
		\item What happens after penetration of the sample?
		\item Attenuation
		\item Conversion to visible light by Scintillator
		\item Detection, recording
	\end{itemize}
\end{frame}

\begin{frame}[label=asdfcurrent]{Reconstructions}
	\centering
	\animategraphics[autoplay,palindrome,height=\imheight,every=\everyframe]{25}{./movies/scan/reconstructions/KP-TNIKWT02_240_reconstructions_of_556_800_px_}{000}{277}
\end{frame}
	
\begin{frame}[label=asdfcurrent]{Reconstructions}
	\begin{itemize}
		\item Based on hundreds of angular views acquired while the object rotates, a computer synthesizes a stack of virtual cross section slices through the object.
		\item Radon Transformation
		\item Filtered back projection
		\item Fan beam reconstruction
		\item Corrections (beam hardening, etc.)
		\item Writing to stack
	\end{itemize}
\end{frame}

\begin{frame}[label=asdfcurrent]{Visualization}
	\begin{tikzpicture}[remember picture,overlay]%
		\node at (current page.center){%
			\animategraphics[autoplay,palindrome,width=\paperwidth,every=\everyframe]{25}{./movies/scan/visualization/lung}{000}{473}%
		};%
	\end{tikzpicture}
\end{frame}

\begin{frame}[label=asdfcurrent]{Visualization}
	\begin{itemize}
		\item Based the reconstructions, a computer synthesizes a three-dimensional view of the scanned sample
	\end{itemize}
\end{frame}

\renewcommand{\imwidth}{1.6\columnwidth}
\begin{frame}[label=asdfcurrent]{Machinery}
	\begin{columns}
		\begin{column}{0.5\linewidth}
			\begin{itemize}
				\item<1-> Hospital CT
				\item<2-> Lab CT
				\item<3-> Desktop CT
				\item<4-> Synchrotron CT
			\end{itemize}
		\end{column}
		\begin{column}{0.5\linewidth}
			\includegraphics<1>[height=0.618\textheight]{./images/24324062640_751e011e1a_o}
			\only<1>{\source{flic.kr/p/D4rbom}{\ccbyncsa}}
			\includegraphics<2>[height=0.618\textheight]{./images/9459311320_516179207a_o}
			\only<2>{\source{flic.kr/p/fpTrGu}{\ccbysa}}
			\includegraphics<3>[width=\imwidth]{./images/1272}
			\only<3>{\source{bruker.com/skyscan1272}{}}
			\includegraphics<4>[width=\imwidth]{./images/4563733710_f632792416_b}
			\only<4>{\source{flic.kr/p/7Xhk2Y}{\ccbync}}
		\end{column}
	\end{columns}
\end{frame}

% FAN BEAM -> BRUKER
% PARALLEL BEAM -> TOMCAT
% HELICAL/SPIRAL CT
% MULTISLICE CT -> HOSPITAL
% SCREENING/IN VIVO/DENSITOMETRY/ETC.

\subsection{Imaging performance}
% RESOLUTION/PIXEL SIZE -> PHANTOM?
% POINT SPREAD FUNCTION
% NOISE
% TEMPORAL RESOLUTION (ONLY FOR HUMAN MEDICAL CT)

\section{Image processing}
\subsection{Image display}
% MIP
% SURFACE RENDERING

\begin{frame}{What to use?}
	\begin{itemize}
		\item ImageJ~\footcite{Schindelin2012}
		\item \faPython%
		\item See \href{https://ilias.unibe.ch/goto_ilias3_unibe_sess_1561468.html}{\emph{Fundamentals of Digital Image Processing} by Guillaume Witz}
	\end{itemize}
\end{frame}

\begin{frame}{Quantitative data}
	\begin{itemize}
		\item Raw numbers instead of just pretty images
		\item Segmentation
		\item Characterization
	\end{itemize}
\end{frame}

\begin{frame}{Big data}
	\begin{itemize}
		\item TOMCAT \SI{2560x2160}{px} @ \SI{1500}{f\per\second}, \eg{} more than \SI{8}{\giga\byte\per\second}
		\item Desktop \uct{} can easily produce more than \SI{100}{\giga\byte} in a day
	\end{itemize}
\end{frame}

\begin{frame}{Reproducible research}
	\begin{itemize}
		\item \faGit%
		\item Jupyter
		\item Script all your things!
		\item Data repositories \(\rightarrow\) Sharing is caring!
	\end{itemize}
\end{frame}

\begin{frame}{\href{https://en.wikipedia.org/wiki/Colophon_(publishing)}{Colophon}}
	\begin{itemize} 
		\item This \textsc{beamer} presentation was crafted in \LaTeX\xspace with the (slightly adapted) \href{http://intern.unibe.ch/dienstleistungen/corporate_design_und_vorlagen/praesentationen/index_ger.html}{template from \emph{Corporate Design und Vorlagen} of the University of Bern}.
		\begin{itemize}
			\item \href{https://github.com/habi/lecture.microtomography/}{Full source code}: git.io/fjpP7
			\item The \LaTeX\xspace code is automatically compiled with a \href{https://github.com/actions}{GitHub action}\footnote{Details on how this works are specified in a \href{https://github.com/habi/latex-test/}{small test repository here}: git.io/JeOOj} to a PDF which you can access here: XXXX
			\item Spotted an error?\\%
				Then please file an \href{https://github.com/habi/lecture.microtomography/issues}{issue} (git.io/fjpPb) or (even better) submit a \href{https://github.com/habi/lecture.microtomography/pulls}{pull request} (git.io/fjpPN).
		\end{itemize}
	\end{itemize}
\end{frame}

\begin{frame}{References}
	% Make the references continuously smaller :)
	%\renewcommand*{\bibfont}{\small}
	%\renewcommand*{\bibfont}{\footnotesize}
	%\renewcommand*{\bibfont}{\scriptsize}
	%\renewcommand*{\bibfont}{\tiny}
	\setbeamertemplate{bibliography item}{\insertbiblabel}
	\printbibliography
\end{frame}

\end{document}

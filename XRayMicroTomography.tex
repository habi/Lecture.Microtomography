% Compile with:
% latexmk -pdf -pvc -interaction=nonstopmode
%\documentclass[aspectratio=169,10pt,draft]{beamer}
\documentclass[aspectratio=169,10pt]{beamer}
%\documentclass[handout,aspectratio=169,10pt]{beamer}
%\documentclass[notes=only,aspectratio=169,10pt]{beamer} % print out the notes

\usetheme{UniBern}

\title{X-ray microtomography}
\author{David Haberthür}
\date{December 20, 2019 | \href{https://ilias.unibe.ch/ilias.php?ref_id=1555744&cmd=infoScreen&cmdClass=ilrepositorygui&cmdNode=y2&baseClass=ilrepositorygui}{9256-HS2019-0: Advanced Microscopy}}

%\includeonlyframes{current}
%then....
%\begin{frame}[label=current]
%\end{frame}

\usepackage[backend=biber,
	style=numeric,
	url=false,
	isbn=true,
	maxbibnames=1,
	maxcitenames=1,
	sorting=none]{biblatex}
	\addbibresource{../../../Documents/library.bib}
\usepackage{standalone}
\usepackage{tikz}
	\usetikzlibrary{spy}
	%\usetikzlibrary{external}
	%\tikzexternalize[prefix=tikz-externals/] % externalize tikz figures into pdf frames, speeds up compilation
	\tikzset{shadowed/.style={preaction={transform canvas={shift={(1pt,-1pt)}},draw=ubRed, thick}}}
\usepackage{shadowtext}  % for the shadowed scalebar
	\shadowoffset{1pt}
	\shadowcolor{ubRed}
\usepackage{pgfplots}
	\pgfplotsset{compat=newest}
\usepackage[detect-all=true,
	range-phrase=--,
	range-units=single,
	binary-units=true,
	per-mode=symbol,
	per-symbol=/]{siunitx}
\usepackage{microtype}	
\usepackage[absolute,overlay]{textpos} %for the \source{} command
\usepackage[missing=master]{gitinfo2} % GitHub Actions don't pull in the commit has, so we just show `Master'
\usepackage{xspace}
\usepackage{ccicons}
\usepackage{animate}
\usepackage{fontawesome5}

% Some often used abbreviations/commands
\newcommand{\everyframe}{75}% use only every nth frame for the movies
\newcommand{\imwidth}{\linewidth}% set global image width
\newcommand{\imheight}{0.618\paperheight}% set global image height
\newlength\imagewidth% needed for scalebars
\newlength\imagescale% needed for scalebars
\newcommand{\uct}{\si{\micro}CT\xspace}% make our life easier
\newcommand{\eg}{e.\,g.\xspace}%
\newcommand{\ie}{i.\,e.\xspace}%

% change tikz font to slide font
% https://tex.stackexchange.com/a/33329/828
%\usepackage[eulergreek]{sansmath}
%	\pgfplotsset{tick label style = {font=\sansmath\sffamily},
%		every axis label = {font=\sansmath\sffamily},
%		legend style = {font=\sansmath\sffamily},
%		label style = {font=\sansmath\sffamily}
%		}

% Globally thicker lines in with tikz
% https://tex.stackexchange.com/a/206769/828
%\tikzset{every picture/.style={semithick}}

% And thicker plots by default
% https://tex.stackexchange.com/a/235439/828
% https://tex.stackexchange.com/q/262486/828
%\pgfplotsset{
%	every axis plot/.append style={semithick},
%	every axis/.append style={semithick}
%	every axis plot post/.append style={every mark/.append style={semithick}}
%}

% stripped-down plot styling
% Based on https://tex.stackexchange.com/a/155210/828
% And then start each plot with `\begin{axis}[tuftelike,'
%\pgfkeys{%
%	/pgfplots/simplified/.style={%
%		tick style={major tick length=0pt},
%%		separate axis lines,
%%		axis x line*=bottom,
%%		axis x line shift=10pt,
%%		xlabel shift=1pt,
%%		axis y line*=left,
%%		axis y line shift=10pt,
%		ymajorgrids=true,
%		axis line style={draw=none},
%%		ylabel shift=1pt
%		}
%	}

% Acknowledge images just below them
% Based on https://tex.stackexchange.com/a/282637/828
\newcommand{\source}[2]{%
	% Print out (short) link under image, with small text
	\raisebox{-1.618ex}{%
		\makebox[0pt][r]{%
			\scriptsize\href{http://#1}{#1} #2%
			}%
		}%
	}%
\newcommand{\sourcecite}[2]{%
	% Cite (an image from) a reference
	\raisebox{-1.618ex}{%
		\makebox[0pt][r]{%
			\scriptsize From \cite{#1}, #2%
			}%
		}%
	}%
\newcommand{\sourcelink}[3]{%
	% Make the source command an \href{link}{text}
	\raisebox{-1.618ex}{%
		\makebox[0pt][r]{%
			\scriptsize\href{http://#1}{#2}, #3%
			}%
		}%
	}%	

% Define us our custom footer
%\defbeamertemplate{footline}{unibe}{%
%	\hspace*{\fill}%
%	v. \href{https://github.com/habi/lecture.microtomography/commit/\gitHash}{\gitAbbrevHash}\xspace|\xspace%
%	p.\xspace\insertframenumber/\inserttotalframenumber%
%	\hspace*{4ex}%
%	\vspace{2pt}%
%}
%\setbeamertemplate{footline}[unibe]

% Define us a custom footer *with* progress bar, based on https://tex.stackexchange.com/a/59749/828
\makeatletter
\def\progressbar@progressbar{} % the progress bar
\newcount\progressbar@tmpcounta% auxiliary counter
\newcount\progressbar@tmpcountb% auxiliary counter
\newdimen\progressbar@pbht %progressbar height
\newdimen\progressbar@pbwd %progressbar width
\newdimen\progressbar@rcircle % radius for the circle
\newdimen\progressbar@tmpdim % auxiliary dimension
\progressbar@pbwd=\linewidth
\progressbar@rcircle=1.5pt
\def\progressbar@progressbar{%
	\progressbar@tmpcounta=\insertframenumber
	\progressbar@tmpcountb=\inserttotalframenumber
	\progressbar@tmpdim=\progressbar@pbwd
	\multiply\progressbar@tmpdim by \progressbar@tmpcounta
	\divide\progressbar@tmpdim by \progressbar@tmpcountb
	\par%
	\begin{tikzpicture}%
		\draw[ubGrey] (0,0) -- ++ (\progressbar@pbwd,0);
		\filldraw[ubGrey] (\the\dimexpr\progressbar@tmpdim-\progressbar@rcircle\relax,.5\progressbar@pbht) circle (\progressbar@rcircle);
	\end{tikzpicture}%
	v. \href{https://github.com/habi/lecture.microtomography/commit/\gitHash}{\gitAbbrevHash}\xspace|\xspace%
	p.\xspace\insertframenumber/\inserttotalframenumber%
%	\hspace*{4ex}%
	\vspace{0.7ex}
	\par%
}
\addtobeamertemplate{footline}{}
{%
	\begin{beamercolorbox}[wd=\paperwidth,center]{white}%
		\progressbar@progressbar%
	\end{beamercolorbox}%
}
\makeatother

% Format bibliography for beamer
% http://tex.stackexchange.com/a/10686/828
\renewbibmacro{in:}{}
% http://tex.stackexchange.com/a/13076/828
\AtEveryBibitem{%
	\clearfield{journaltitle}
	\clearfield{pages}
	\clearfield{volume}
	\clearfield{number}
	\clearname{editor}
	\clearfield{issn}
	\clearfield{year}
}
% No parentheses around the (now empty) year: https://tex.stackexchange.com/a/147537/828
\renewcommand{\bibopenparen}{\addcomma\addspace}
\renewcommand{\bibcloseparen}{\addcomma\addspace}

% Redefine \footcite based on https://tex.stackexchange.com/a/453528/828
\DeclareCiteCommand{\footcite}[\mkbibfootnote]{%
	\usebibmacro{prenote}}{%
		\printnames[family-given]{labelname}%
		\newunit%
		\printfield{doi}%
		\newunit%
		\printlabeldateextra%
	}{\addsemicolon\space}{%
		\usebibmacro{postnote}%
	}%

% References as footnotes at the bottom of the slides
% https://tex.stackexchange.com/a/368760/828
\makeatletter
\renewcommand\@makefnmark{\xspace\hbox{\usebeamercolor[fg]{footnote mark}\usebeamerfont*{footnote mark}[\@thefnmark]}}
\renewcommand\@makefntext[1]{\tiny{\usebeamercolor[fg]{footnote mark}\usebeamerfont*{footnote mark}[\@thefnmark]}\enspace\usebeamerfont*{footnote} #1}
\makeatother

% Show current section at begin of sections
%\AtBeginSection[]{
%	\begin{frame}{Outline}
%		\small \tableofcontents[currentsection,currentsubsection,hideothersubsections]
%	\end{frame}
%}

% Slide transiton
%\addtobeamertemplate{background canvas}{\transfade[duration=0.5]}{}

% open in fullscreen
%\hypersetup{pdfpagemode=FullScreen}

% Move the text down a bit
% THIS IS A BIG HACK, IT SHOULD BE FIXED IN THE TEMPLATE
\addtobeamertemplate{frametitle}{}{\vspace*{0.75ex}}

\begin{document}
% No footline on the title page
% http://tex.stackexchange.com/a/18829/828 helps us to achieve that
{%
	\setbeamertemplate{footline}{}%
	\begin{frame}%
		\maketitle
	\end{frame}%
}

% Alignment frames to test the ugly text-block-movement hack above
%\begin{frame}[label=current]{Alignment frame}
%	\begin{tikzpicture}%
%		\def\cut{3}%
%		\draw [|-|,ultra thick] (0,\cut) -- (0,\textheight-\cut);%
%		\draw [<->,ultra thick] (0.5\textwidth,\cut) -- (0.5\textwidth,\textheight-\cut);%
%		\draw [|-|,ultra thick] (\textwidth,\cut) -- (\textwidth,\textheight-\cut);%
%	\end{tikzpicture}%
%\end{frame}
%
%\begin{frame}[allowframebreaks,label=current]{Alignment frame II}
%	\lipsum[1-50]
%\end{frame}

\begin{frame}{Hello!}
	\begin{itemize}
		\item Office \href{http://osm.org/go/0CZwlGp3A?m}{B311} | \href{mailto:haberthuer@ana.unibe.ch?subject=Feedback\%20from\%20the\%20(micro)-tomography\%20lecture}{haberthuer@ana.unibe.ch}
		\item Master in Physics, then \href{https://boris.unibe.ch/2619/}{PhD in high resolution imaging of the lung} at the Institute of Anatomy
		\item Post-Doc at the \href{https://www.psi.ch/sls/tomcat/}{TOMCAT beamline} of the \href{https://www.psi.ch/sls/}{Swiss Light Source} at the \href{https://www.psi.ch/}{Paul Scherrer Institute}
		\item Post-Doc at the \href{https://ana.unibe.ch}{Institute of Anatomy} in the \uct-group
		\begin{itemize}
			\item Ruslan Hlushchuk, David Haberthür, Oleksiy-Zakhar Khoma, Fluri Wieland, Carlos Correa Shokiche
		\end{itemize}			
		\item Biomedical research
		\begin{itemize}
			\item microangioCT~\footcite{Hlushchuk2018}: Tumor vasculature, angiogenesis in the heart, musculature and bones
			\item Cancer research: Melanoma
			\item Lung imaging: Tumor detection and classification
			\item Physiology: Zebrafish musculature and gills~\footcite{Messerli2019}
			\item SkyScan 1172 \& 1272
		\end{itemize}
	\end{itemize}		
\end{frame}

\begin{frame}{Contents}
%	\begin{multicols}{2}
	\tableofcontents
%	\end{multicols}
\end{frame}

\section{Biomedical imaging}
\begin{frame}[c]{Biomedical imaging}
	\begin{columns}
		\begin{column}{0.4\linewidth}
			\begin{itemize}
				\item<1-> Medical research
				\item<2->(Small) Biological samples
				\item<3-> Non-destructive insights into the samples

			\end{itemize}
		\end{column}	
		\begin{column}{0.6\linewidth}
			\centering
			\only<1>{%
				\includegraphics[height=0.618\textheight]{./images/Sagittal_brain_MRI}
				\source{w.wiki/7g4}{\ccbysa}
			}
			\only<3>{%
				%\animategraphics[palindrome,autoplay,width=\paperwidth,every=\everyframe]{25}{./movies/mouse_skull/mouse_skull}{000}{470}%
				%\par%
				{%
				% Do not externalize bitmap animations
				% https://tex.stackexchange.com/a/516425/828
				%\tikzset{external/export=false}%
				\begin{tikzpicture}[remember picture,overlay]%
					\node at (current page.center){%
						\animategraphics[palindrome,autoplay,width=\paperwidth,every=\everyframe]{25}{./movies/mouse_skull/mouse_skull}{000}{470}%
					};%
				\end{tikzpicture}%
				}
			}
		\end{column}
	\end{columns}
\end{frame}

\section{Imaging}
\begin{frame}{Wavelength \& Scale}
	\centering
	\includegraphics[height=0.618\textheight]{./images/2000px-Electromagnetic_spectrum_with_sources}
	\source{w.wiki/7fz}{\ccbysa}
\end{frame}

\begin{frame}{Imaging methods}
	\begin{itemize}
		\item Light microscopy: see \href{https://ilias.unibe.ch/goto_ilias3_unibe_sess_1555739.html}{lecture of Nadia Mercader Huber}
		\item X-ray imaging
		\item Electron microscopy: see lectures on
			\href{https://ilias.unibe.ch/goto_ilias3_unibe_sess_1555738.html}{Transmission Electron Microscopy by Dimitri Vanhecke},
			\href{https://ilias.unibe.ch/goto_ilias3_unibe_sess_1555742.html}{Scanning Electron Microscopy by Michael Stoffel} and
			\href{https://ilias.unibe.ch/goto_ilias3_unibe_sess_1555743.html}{Cryoelectron Microscopy \& Serial Block Face SEM by Ioan Iacovache}.
	\end{itemize}
\end{frame}

\subsection{Tomography}
\begin{frame}{CT-Scanner}
	\centering
	\animategraphics[palindrome,autoplay,height=0.618\textheight,every=\everyframe]{25}{./movies/ct-scanner/ct-scanner0}{001}{480}
	\source{youtu.be/2CWpZKuy-NE}{}
	\note{From \url{https://www.bruker.com/products/microtomography/micro-ct-for-sample-scanning/x-ray-micro-ct-microtomography.html}: 
		Micro computed tomography or micro-CT is x-ray imaging in 3D, by the same method used in hospital CT (or CAT) scans, but on a small scale with massively increased resolution.
		It really represents 3D microscopy, where very fine scale internal structure of objects is imaged non-destructively.
		No sample preparation, no staining, no thin slicing - a single scan will image your sample's complete internal 3D structure at high resolution, plus you get your intact sample back at the end!}
\end{frame}

\begin{frame}{What is happening?}
	\centering
	\includegraphics[height=0.618\textheight]{./images/3D_Computed_Tomography}
	\source{w.wiki/7g3}{\ccbysa}
\end{frame}

\subsection{X-ray production}
\begin{frame}{X-ray generation}
	\begin{itemize}
		\item How are x-rays generated
		\item Why do we need them
	\end{itemize}
\end{frame}

\subsection{Interaction of x-rays with matter}
\begin{frame}[label=current]{X-ray generation}
	\begin{itemize}
		\item Photoelectric absorption (\(\tau\)) is strongly dependent on the atomic number \(Z\) of the absorbing material (
	\(\tau\propto\frac{Z^4}{E^{3.5}}\)
	\end{itemize}
\end{frame}
\note{From \emph{https://radiopaedia.org/articles/photoelectric-effect}{Radiopaedia.org}: Therefore if Z doubles, PEA will increase by a factor of 16 (2^4 = 8), and if E doubles PEA will reduce by a factor of 16.
Small changes in Z and E can therefore significantly affect PEA.
This has practical implications in the field of radiation protection and is the reason why materials with a high Z such as lead (Z = 82) are useful shielding materials.
The dependence of PEA on Z and E means that it is the major contributor to beam attenuation up to approximately 30 keV when human tissues (Z = 7.4) are irradiated.
At beam energies above this, the Compton effect predominates.}

% TODO: Insert table from Joliens presentation -> let the students see why the bone is so white in x-ray images

\subsection{History}
\begin{frame}{History}
	\begin{columns}
		\begin{column}{0.48\linewidth}
			\begin{itemize}
				\item Some history is found in~\cite{Cormack1963,Hsieh2003}
				\item<2-> First, second and third generation of scanners
			\end{itemize}
		\end{column}
		\begin{column}{0.48\linewidth}
			\centering
			\includegraphics<2>[width=\imwidth]{./images/History_Generation1}%
			\only<2>{\sourcecite{Hsieh2003}{Figure 1.12}}%
			\includegraphics<3>[width=\imwidth]{./images/History_Generation2}%
			\only<3>{\sourcecite{Hsieh2003}{Figure 1.13}}%
			\includegraphics<4>[width=\imwidth]{./images/History_Generation3}%
			\only<4>{\sourcecite{Hsieh2003}{Figure 1.14}}%
		\end{column}
	\end{columns}
\end{frame}

\renewcommand{\imwidth}{\columnwidth}
\begin{frame}{Machinery}
	\begin{columns}
		\begin{column}{0.4\linewidth}
			\begin{itemize}
				\item<1-> Hospital CT
				\begin{itemize}
					\item Voxel size around \SI{0.5}{\milli\meter}
				\end{itemize}
				\item<2-> Lab/Desktop CT
				\begin{itemize}
					\item Voxel size around \SI{7}{\micro\meter} (\emph{in vivo}) or \SI{0.5}{\micro\meter} (\emph{ex vivo})
				\end{itemize}
				\item<4-> Synchrotron CT
				\begin{itemize}
					\item Voxel size down to \href{https://www.psi.ch/en/sls/tomcat/detectors}{\SI{160}{\nano\meter}}
				\end{itemize}
			\end{itemize}
		\end{column}
		\begin{column}{0.6\linewidth}
			\centering
			\includegraphics<1>[height=\imheight]{./images/24324062640_751e011e1a_o}%
			\only<1>{\source{flic.kr/p/D4rbom}{\ccbyncsa}}
			\includegraphics<2>[width=\imheight]{./images/9459311320_516179207a_o}%
			\only<2>{\source{flic.kr/p/fpTrGu}{\ccbysa}}
			\includegraphics<3>[width=\imwidth]{./images/1272}%
			\only<3>{\source{bruker.com/skyscan1272}{}}
			\includegraphics<4>[width=\imwidth]{./images/4563733710_f632792416_b}%
			\only<4>{\source{flic.kr/p/7Xhk2Y}{\ccbync}}
		\end{column}
	\end{columns}
\end{frame}

\begin{frame}{Machinery I}
Independent on the machine, technically they are all a simple combination of
\begin{itemize}
	\item an x-ray source
	\item a sample
	\item a detector
\end{itemize}
\end{frame}

\begin{frame}{Machinery II}
	\begin{columns}
		\hfill
		\begin{column}{0.4\linewidth}
			%\documentclass{beamer}% For self-contained compilation
\documentclass[tikz]{standalone}% For inclusion in the presentation
% Draw the setup where the source and detector move, e.g. classic CT
% With help from https://tex.stackexchange.com/q/515519/828
\usepackage{animate}
\usepackage{tikz}
%	\usetikzlibrary{external}
%	\tikzexternalize
%	\tikzsetnextfilename{classicCT}
\usepackage{fontawesome5}
\usepackage{ifthen}
\ifthenelse{\isundefined{\everyframe}}{%
	% If we're compiling this file via \input, then these variables are already defined.
	% In the other case, we need to define them
	\newcommand{\everyframe}{5}
	\definecolor{ubRed}{HTML}{e4003c}%
	\definecolor{ubGrey}{HTML}{646363}%
	% split complementary images from https://www.sessions.edu/color-calculator/
	\definecolor{ubRedComplementary}{HTML}{2EE600}
	}{}
\begin{document}
\begin{animateinline}[loop,every=\everyframe]{25}
	\multiframe{90}{n=1+4}{%
		\begin{tikzpicture}[scale=1.618]
			\pgfdeclarelayer{background}
			\pgfsetlayers{background,main}
			\mode<beamer>{%
				%Help lines, to force same size of images
				\begin{pgfonlayer}{background}
					\draw[ubGrey,transparent] (-2.25,0) -- (2.25,0);
					\draw[ubGrey,transparent] (0,-2.25) -- (0,2.25);
					\draw[ubGrey,transparent,help lines,step=1cm,ultra thin] (-2.45,-2.45) grid (2.45,2.45);
				\end{pgfonlayer}
			}%
			% Stuff that stays put
			\node[ubRedComplementary] at (0,0) (sample) {\fontsize{64}{60}\selectfont \faUser};% Large patient: https://tex.stackexchange.com/a/716/828
			% Stuff that moves
			\begin{scope}[rotate around={\n:(sample)}]
				% Rotation arc
				\draw[->, thick,line cap=rect] (1.5,0) arc [start angle=0, end angle=180, radius=1.5];
				\draw[->, thick,line cap=rect] (-1.5,0) arc [start angle=-180, end angle=0, radius=1.5];
				% Source
				\fill[red] (-0.25,1.5) rectangle node (source) {} +(0.5,0.5);
				\draw[fill=yellow] (0,1.75) circle (0.2);
				\node at (0,1.735) (radiation) {\faRadiation};
				% Detector and detector edges
				\fill[ubGrey] (-0.5,-1.75) rectangle node (detector) {} +(1,0.25);
				\coordinate (dl) at (-0.45,-1.75);
				\coordinate (dr) at (0.45,-1.75);
				% X-ray cone
				\begin{pgfonlayer}{background}
					\fill[ubGrey, semitransparent] (source.center) -- (dl) -- (dr) -- cycle;
				\end{pgfonlayer}
			\end{scope}
		\end{tikzpicture}
	}
\end{animateinline}
\end{document}

		\end{column}
		\hfill		
		\begin{column}{0.4\linewidth}
			\documentclass{standalone}%
% Draw the setup where the source and detector move, e.g. classic CT
% With help from https://tex.stackexchange.com/q/515519/828
\usepackage{fontawesome5}
\usepackage{ifthen}
\ifthenelse{\isundefined{\everyframe}}{%
	% If we're compiling this file via \input, then we already defined some things
	% In the other case, we need to define them
	\usepackage{tikz}
	\usepackage{animate}
	\newcommand{\everyframe}{5}
	\definecolor{ubRed}{HTML}{e4003c}%
	\definecolor{ubGrey}{HTML}{646363}%
	% split complementary images from https://www.sessions.edu/color-calculator/
	\definecolor{ubRedComplementary}{HTML}{2EE600}
	}{}
\begin{document}
\begin{animateinline}[loop,every=\everyframe]{25}
	\multiframe{90}{n=1+4}{%
		\begin{tikzpicture}[scale=1.25]
			\pgfdeclarelayer{background}
			\pgfsetlayers{background,main}
			%Help lines used to setup the animation (set to semitransparent), drawing them transparent in the presentation forces a consistent size
			\begin{pgfonlayer}{background}
				\draw[ubGrey,transparent,help lines,step=5mm] (-0.75,-1.25) grid (0.75,1.5);
			\end{pgfonlayer}
			% Stuff that stays put
			% Source
			\fill[ubRed] (-0.25,1) rectangle node (source) {} +(0.5,0.5);
			\draw[fill=yellow] (0,1.25) circle (0.2);
			\node at (0,1.235) (radiation) {\small\faRadiation};
			% Detector and detector edges
			\fill[gray] (-0.5,-1.25) rectangle node (detector) {} +(1,0.25);
			\coordinate (dl) at (-0.45,-1);
			\coordinate (dr) at (0.45,-1);
			% X-ray cone
			\begin{pgfonlayer}{background}
				\fill[gray,semitransparent] (source.center) -- (dl) -- (dr) -- cycle;
			\end{pgfonlayer}
			% Stuff that moves
				\begin{scope}[rotate around={\n:(0,0)}]
				% Rotation arc
				\draw[->, thick,line cap=rect] (0.618,0) arc [start angle=0, end angle=180, radius=0.618];
				\draw[->, thick,line cap=rect] (-0.618,0) arc [start angle=-180, end angle=0, radius=0.618];
				% Sample
				\node[ubRedComplementary] at (0,0) (sample) {\rotatebox{\n}{\normalsize\faFish}};
				\end{scope}
		\end{tikzpicture}
	}
\end{animateinline}
\end{document}

		\end{column}
		\hfill
	\end{columns}
\end{frame}

% FAN BEAM -> BRUKER
% PARALLEL BEAM -> TOMCAT
% HELICAL/SPIRAL CT
% MULTISLICE CT -> HOSPITAL
% SCREENING/IN VIVO/DENSITOMETRY/ETC.

\subsection{A scan, from start to finish}
\begin{frame}{Preparation}
	\begin{itemize}
		\item Study design
		\item Sample preparation
	\end{itemize}
\end{frame}

\renewcommand{\imwidth}{\columnwidth}
\begin{frame}[label=current]{Why \uct?}
	\begin{columns}
		\hfill
		\begin{column}{0.48\linewidth}
			% https://www.cancerimagingarchive.net/nbia-search/?saved-cart=nbia-76761575299081509 
			\only<1-4>{%
				\pgfmathsetlength{\imagewidth}{\imwidth}%
				\pgfmathsetlength{\imagescale}{\imagewidth/512}%
				\def\x{316}% scalebar-x starting at golden ratio of image width of 512px = 316
				\def\y{361}% scalebar-y at 90% of image height of 401px = 361
				\begin{tikzpicture}[x=\imagescale,y=-\imagescale]
					\node[anchor=north west, inner sep=0pt, outer sep=0pt] at (0,0) {\includegraphics[width=\imagewidth]{./images/comparison/MAX_human}};
					% 512.000px = 250.0096mm -> 100px = 48830.000um -> 1.024px = 500um, 0.205px = 100um
					%\draw[|-|,blue,thick] (0,200) -- (512,200) node [sloped,midway,above,fill=white,semitransparent,text opacity=1] {\SI{250.0096}{\milli\meter} (512px) TEMPORARY!};
					\draw[|-|,white,thick,shadowed] (\x,\y) -- (\x+102.4,\y) node [midway,above] {\shadowtext{\SI{5}{\centi\meter}}};
				\end{tikzpicture}%
			}%
			\only<5>{%
				\pgfmathsetlength{\imagewidth}{\imwidth}%
				\pgfmathsetlength{\imagescale}{\imagewidth/512}%
				\def\x{316}% scalebar-x starting at golden ratio of image width of 512px = 316
				\def\y{361}% scalebar-y at 90% of image height of 401px = 361
				\def\mag{4}% magnification of inset
				\def\size{75}% size of inset
				\begin{tikzpicture}[x=\imagescale,y=-\imagescale,spy using outlines={rectangle,magnification=\mag,size=\size,connect spies}]
					\node[anchor=north west, inner sep=0pt, outer sep=0pt] at (0,0) {\includegraphics[width=\imagewidth]{./images/comparison/MAX_human}};
					\spy [red] on (102,342) in node at (256,201) [anchor=center];
					% 512.000px = 250.0096mm -> 100px = 48830.000um -> 1.024px = 500um, 0.205px = 100um
					\draw[|-|,white,thick,shadowed] (\x,\y) -- (\x+102.4,\y) node [midway,above] {\shadowtext{\SI{5}{\centi\meter}}};
				\end{tikzpicture}%
			}%
			\sourcecite{Clark2013}{Subject \emph{C3L-02465}}
		\end{column}
		\hfill		
		\begin{column}{0.48\linewidth}
			\only<1>{%
				\pgfmathsetlength{\imagewidth}{\imwidth}%
				\pgfmathsetlength{\imagescale}{\imagewidth/3295}%
				\def\x{2036}% scalebar-x starting at golden ratio of image width of 3295px = 2036
				\def\y{1343}% scalebar-y at 90% of image height of 1492px = 1343
				\begin{tikzpicture}[x=\imagescale,y=-\imagescale]
					\node[anchor=north west, inner sep=0pt, outer sep=0pt] at (0,0) {\includegraphics[width=\imagewidth]{./images/comparison/MAX_mouse}};
					% 3295.000px = 26.2282mm -> 100px = 796.000um -> 62.814px = 500um, 12.563px = 100um
					%\draw[|-|,blue,thick] (0,746) -- (3295,746) node [sloped,midway,above,fill=white,semitransparent,text opacity=1] {\SI{26.2282}{\milli\meter} (3295px) TEMPORARY!};
					\draw[|-|,white,thick,shadowed] (\x,\y) -- (\x+628.14,\y) node [midway,above] {\shadowtext{\SI{5}{\milli\meter}}};
				\end{tikzpicture}%
				}
			\renewcommand{\imwidth}{0.1\columnwidth}%
			\only<2>{%
				\centering
				\pgfmathsetlength{\imagewidth}{\imwidth}%
				\pgfmathsetlength{\imagescale}{\imagewidth/54}%
				\def\x{33}% scalebar-x starting at golden ratio of image width of 54px = 33
				\def\y{22}% scalebar-y at 90% of image height of 24px = 22
				\def\mag{4}% magnification of inset
				\def\size{75}% size of inset
				\begin{tikzpicture}[x=\imagescale,y=-\imagescale]
					\node[anchor=north west, inner sep=0pt, outer sep=0pt] at (0,0) {\includegraphics[width=\imagewidth]{./images/comparison/MAX_mouse_488umppx}};
					% 54.000px = 26.3682mm -> 100px = 48830.000um -> 1.024px = 500um, 0.205px = 100um
					%\draw[|-|,blue,thick] (0,12) -- (54,12) node [sloped,midway,above,fill=white,semitransparent,text opacity=1] {\SI{26.3682}{\milli\meter} (54px) TEMPORARY!};
					\draw[|-|,white,thick,shadowed] (\x,\y) -- (\x+102.4,\y) node [midway,above] {\shadowtext{\SI{5}{\centi\meter}}};
					%\draw[color=red, anchor=south west] (0,24) node [fill=white, semitransparent] {Legend} node {Legend};
\end{tikzpicture}%				
					}
			\renewcommand{\imwidth}{\columnwidth}
			\only<3>{%
				\centering
				\pgfmathsetlength{\imagewidth}{\imwidth}%
				\pgfmathsetlength{\imagescale}{\imagewidth/54}%
				\def\x{33}% scalebar-x starting at golden ratio of image width of 54px = 33
				\def\y{22}% scalebar-y at 90% of image height of 24px = 22
				\def\mag{4}% magnification of inset
				\def\size{75}% size of inset
				\begin{tikzpicture}[x=\imagescale,y=-\imagescale]
					\node[anchor=north west, inner sep=0pt, outer sep=0pt] at (0,0) {\includegraphics[width=\imagewidth]{./images/comparison/MAX_mouse_488umppx}};
					% 54.000px = 26.3682mm -> 100px = 48830.000um -> 1.024px = 500um, 0.205px = 100um
					%\draw[|-|,blue,thick] (0,12) -- (54,12) node [sloped,midway,above,fill=white,semitransparent,text opacity=1] {\SI{26.3682}{\milli\meter} (54px) TEMPORARY!};
					\draw[|-|,white,thick,shadowed] (\x,\y) -- (\x+10.24,\y) node [midway,above] {\shadowtext{\SI{5}{\milli\meter}}};
					%\draw[color=red, anchor=south west] (0,24) node [fill=white, semitransparent] {Legend} node {Legend};
\end{tikzpicture}%				
					}%
			\only<4>{%
				\pgfmathsetlength{\imagewidth}{\imwidth}%
				\pgfmathsetlength{\imagescale}{\imagewidth/3295}%
				\def\x{2036}% scalebar-x starting at golden ratio of image width of 3295px = 2036
				\def\y{1343}% scalebar-y at 90% of image height of 1492px = 1343
				\def\mag{4}% magnification of inset
				\def\size{75}% size of inset
				\begin{tikzpicture}[x=\imagescale,y=-\imagescale,spy using outlines={rectangle,magnification=\mag,size=\size,connect spies}]
					\node[anchor=north west, inner sep=0pt, outer sep=0pt] at (0,0) {\includegraphics[width=\imagewidth]{./images/comparison/MAX_mouse}};
					% 3295.000px = 26.2282mm -> 100px = 796.000um -> 62.814px = 500um, 12.563px = 100um
					\draw[|-|,white,thick,shadowed] (\x,\y) -- (\x+628.14,\y) node [midway,above] {\shadowtext{\SI{5}{\milli\meter}}};
				\end{tikzpicture}%
			}%					
			\only<5>{%
				\pgfmathsetlength{\imagewidth}{\imwidth}%
				\pgfmathsetlength{\imagescale}{\imagewidth/3295}%
				\def\x{2036}% scalebar-x starting at golden ratio of image width of 3295px = 2036
				\def\y{1343}% scalebar-y at 90% of image height of 1492px = 1343
				\def\mag{4}% magnification of inset
				\def\size{75}% size of inset
				\begin{tikzpicture}[x=\imagescale,y=-\imagescale,spy using outlines={rectangle,magnification=\mag,size=\size,connect spies}]
					\node[anchor=north west, inner sep=0pt, outer sep=0pt] at (0,0) {\includegraphics[width=\imagewidth]{./images/comparison/MAX_mouse}};
					\spy [red] on (352,1116) in node at (1648,746) [anchor=center];
					% 3295.000px = 26.2282mm -> 100px = 796.000um -> 62.814px = 500um, 12.563px = 100um
					\draw[|-|,white,thick,shadowed] (\x,\y) -- (\x+628.14,\y) node [midway,above] {\shadowtext{\SI{5}{\milli\meter}}};
				\end{tikzpicture}%
			}%
		\end{column}
		\hfill
	\end{columns}
\end{frame}

\renewcommand{\imwidth}{0.48\linewidth}
\begin{frame}{Projections}
	\includegraphics[width=\imwidth]{./images/14_Physics_updated}%
	\sourcelink{ilias.unibe.ch/goto\_ilias3\_unibe\_sess\_1555733.html}{\emph{Laws of Physics for Microscopists} by Michael Jaeger}{Slide 14}%
	\hfill
	\includegraphics[width=\imwidth]{./images/15_Physics_updated}%
	\sourcelink{ilias.unibe.ch/goto\_ilias3\_unibe\_sess\_1555733.html}{\emph{Laws of Physics for Microscopists} by Michael Jaeger}{Slide 15}%
\end{frame}

\begin{frame}{Projections}
	\centering
	% Movie frames generated with https://github.com/habi/Lecture.Microtomography/blob/master/Notebooks/FromProjectionsToReconstructions.ipynb
	\animategraphics[autoplay,loop,height=\imheight,every=\everyframe]{25}{./movies/scan/projections/KP-TNIKWT02_240_projections_of_940_800_px_}{000}{313}
\end{frame}

\begin{frame}{Projections}
	\begin{itemize}
		\item A (micro-focus) x-ray source illuminates the object
		\item A planar x-ray detector collects magnified projection images.
		\item What happens after penetration of the sample?
		\item Attenuation
		\item Conversion to visible light by Scintillator
		\item Detection, recording
	\end{itemize}
\end{frame}

\begin{frame}{Reconstructions}
	\centering
	% Movie frames generated with https://github.com/habi/Lecture.Microtomography/blob/master/Notebooks/FromProjectionsToReconstructions.ipynb
	\animategraphics[autoplay,palindrome,height=\imheight,every=\everyframe]{25}{./movies/scan/reconstructions/KP-TNIKWT02_240_reconstructions_of_556_800_px_}{000}{277}
\end{frame}
	
\begin{frame}{Reconstructions}
	\begin{itemize}
		\item Based on hundreds of angular views acquired while the object rotates, a computer synthesizes a stack of virtual cross section slices through the object.
		\item Radon Transformation
		\item Filtered back projection
		\item Fan beam reconstruction
		\item Corrections (beam hardening, etc.)
		\item Writing to stack
	\end{itemize}
\end{frame}

\begin{frame}{Visualization}
	\begin{tikzpicture}[remember picture,overlay]%
	\node at (current page.center){%
		\animategraphics[autoplay,palindrome,width=\paperwidth,every=\everyframe]{25}{./movies/scan/visualization/lung}{000}{473}%
		};%
	\end{tikzpicture}%
\end{frame}

\begin{frame}{Visualization}
	\begin{itemize}
		\item Based the on reconstructions, a computer synthesizes a three-dimensional view of the scanned sample
	\end{itemize}
\end{frame}

\subsection{Examples}

% TODO: Zebrafish
% TODO: Bone/teeth with screw -> Fluri
% TODO: Osteocytes in bone
% TODO: Toothpick


\subsection{Imaging performance}
% RESOLUTION/PIXEL SIZE -> PHANTOM?
% POINT SPREAD FUNCTION
% NOISE
% TEMPORAL RESOLUTION (ONLY FOR HUMAN MEDICAL CT)

\section{Image processing}
\subsection{Image display}
% MIP
% SURFACE RENDERING

\begin{frame}{What to use?}
	\begin{itemize}
		\item ImageJ~\footcite{Schindelin2012}
		\item \faPython%
		\item See \href{https://ilias.unibe.ch/goto_ilias3_unibe_sess_1561468.html}{\emph{Fundamentals of Digital Image Processing} by Guillaume Witz}
	\end{itemize}
\end{frame}

\begin{frame}{Quantitative data}
	\begin{itemize}
		\item Raw numbers instead of just pretty images
		\item Segmentation
		\item Characterization
	\end{itemize}
\end{frame}

\begin{frame}{Reproducible research}
	\begin{itemize}
		\item \faGit%
		\item Jupyter
		\item Script all your things!
		\item Data repositories \(\rightarrow\) Sharing is caring!
	\end{itemize}
\end{frame}

\begin{frame}{\href{https://en.wikipedia.org/wiki/Colophon_(publishing)}{Colophon}}
	\begin{itemize} 
		\item This \textsc{beamer} presentation was crafted in \LaTeX\xspace with the (slightly adapted) \href{http://intern.unibe.ch/dienstleistungen/corporate_design_und_vorlagen/praesentationen/index_ger.html}{template from \emph{Corporate Design und Vorlagen} of the University of Bern}.
		\begin{itemize}
			\item \href{https://github.com/habi/lecture.microtomography/}{Full source code}: git.io/fjpP7
			\item The \LaTeX\xspace code is automatically compiled with a \href{https://github.com/actions}{GitHub action}\footnote{Details on how this works are specified in a \href{https://github.com/habi/latex-test/}{small test repository here}: git.io/JeOOj} to a \href{https://habi.github.io/Lecture.Microtomography/XRayMicroTomography.pdf}{PDF} which you can access here: git.io/JeMjP
			\item Spotted an error?\\%
				Then please file an \href{https://github.com/habi/lecture.microtomography/issues}{issue} (git.io/fjpPb) or (even better) submit a \href{https://github.com/habi/lecture.microtomography/pulls}{pull request} (git.io/fjpPN).
		\end{itemize}
	\end{itemize}
\end{frame}

\begin{frame}{References}
	% Make the references continuously smaller :)
	%\renewcommand*{\bibfont}{\small}
	%\renewcommand*{\bibfont}{\footnotesize}
	%\renewcommand*{\bibfont}{\scriptsize}
	%\renewcommand*{\bibfont}{\tiny}
	\setbeamertemplate{bibliography item}{\insertbiblabel}
	\printbibliography
\end{frame}

\end{document}

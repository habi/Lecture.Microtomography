%\documentclass{beamer}% For self-contained compilation
\documentclass[tikz]{standalone}% For inclusion in the presentation
% Draw the setup where the only the sample moves, e.g. microCT
% Essentially just a copy of classicCT in the same folder
\usepackage{animate}
\usepackage{tikz}
%	\usetikzlibrary{external}
%	\tikzexternalize
%	\tikzsetnextfilename{microCT}
\usepackage{fontawesome5}
\usepackage{ifthen}
\ifthenelse{\isundefined{\everyframe}}{%
	% If we're compiling this file via \input, then these variables are already defined.
	% In the other case, we need to define them
	\newcommand{\everyframe}{5}
	\definecolor{ubRed}{HTML}{e4003c}%
	\definecolor{ubGrey}{HTML}{646363}%
	% split complementary images from https://www.sessions.edu/color-calculator/
	\definecolor{ubRedComplementary}{HTML}{2EE600}
	}{}
\begin{document}
\begin{animateinline}[loop,every=\everyframe]{25}
	\multiframe{90}{n=1+4}{%
		\begin{tikzpicture}[scale=1.618]
			\pgfdeclarelayer{background}
			\pgfsetlayers{background,main}
			\mode<beamer>{%
				%Help lines, to force same size of images
				\begin{pgfonlayer}{background}
					\draw[ubGrey,transparent] (-2.25,0) -- (2.25,0);
					\draw[ubGrey,transparent] (0,-2.25) -- (0,2.25);
					\draw[ubGrey,transparent,help lines,step=1cm,ultra thin] (-2.45,-2.45) grid (2.45,2.45);
				\end{pgfonlayer}
			}%
			% Stuff that stays put
			% Source
			\fill[red] (-0.25,1) rectangle node (source) {} +(0.5,0.5);
			\draw[fill=yellow] (0,1.25) circle (0.2);
			\node at (0,1.235) (radiation) {\faRadiation};
			% Detector and detector edges
			\fill[ubGrey] (-0.5,-1.25) rectangle node (detector) {} +(1,0.25);
			\coordinate (dl) at (-0.45,-1);
			\coordinate (dr) at (0.45,-1);
			% X-ray cone
			\begin{pgfonlayer}{background}
				\fill[ubGrey,semitransparent] (source.center) -- (dl) -- (dr) -- cycle;
			\end{pgfonlayer}
			% Stuff that moves
				\begin{scope}[rotate around={\n:(0,0)}]
				% Rotation arc
				\draw[->, thick,line cap=rect] (0.5,0) arc [start angle=0, end angle=180, radius=0.5];
				\draw[->, thick,line cap=rect] (-0.5,0) arc [start angle=-180, end angle=0, radius=0.5];
				% Sample
				\node[ubRedComplementary] at (0,0) (sample) {\rotatebox{\n}{\normalsize\faFish}};
				\end{scope}
		\end{tikzpicture}
	}
\end{animateinline}
\end{document}

\documentclass{standalone}
% Draw the setup where the source and detector move, e.g. classic CT
% With help from https://tex.stackexchange.com/q/515519/828\usepackage{animate}
\usepackage{animate}
\usepackage{tikz}
\usetikzlibrary{external}
\tikzexternalize
\tikzsetnextfilename{classicCT-animation}
\usepackage{ifthen}
\ifthenelse{\isundefined{\everyframe}}{%
	% If we're running this file via \input, then \everyframe is already defined
	\newcommand{\everyframe}{1}
	}{}
\begin{document}
\begin{animateinline}[autoplay,every=\everyframe,loop]{25}
	\multiframe{180}{n=1+2}{%
		\begin{tikzpicture}[scale=2]
			\pgfdeclarelayer{background}
			\pgfsetlayers{background,main}
			%Help lines
			\draw[<->] (-1.6,0) -- (1.6,0);
			\draw[<->] (0,-1.6) -- (0,1.6);
			\draw[help lines,step=0.5cm,ultra thin] (-1.45,-1.45) grid (1.45,1.45);
			% Stuff that stays put
			\fill[green] (0,0) ellipse (0.618/2 and 0.618/3) node (sample) [black,opacity=0,text opacity=1] {Sample};
			% Stuff that moves
			\begin{scope}[rotate around={\n:(sample)}]
				% Rotation arc
				\draw[->, thick,line cap=rect] (1,0) arc [start angle=0, end angle=180, radius=1];
				\draw[->, thick,line cap=rect] (-1,0) arc [start angle=-180, end angle=0, radius=1];
				% Source
				\fill[red] (-0.25,1) rectangle node (source) [black,opacity=0, text opacity=1] {X-ray} +(0.5,0.5);
				% Detector and detector edges
				\fill[blue,fill] (-0.5,-1.25) rectangle node (detector) [black,fill=white,semitransparent,text opacity=1] {Detector} +(1,0.25);
				\coordinate (dl) at (-0.45,-1);
				\coordinate (dr) at (0.45,-1);
				% X-ray cone
				\begin{pgfonlayer}{background}
					\fill[gray,semitransparent] (source.center) -- (dl) -- (dr) -- cycle;
				\end{pgfonlayer}
			\end{scope}
		\end{tikzpicture}
	}
\end{animateinline}
\end{document}

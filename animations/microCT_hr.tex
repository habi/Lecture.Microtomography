\documentclass{standalone}
% Draw the setup where the only the sample moves, e.g. microCT
% For the case of small samples close to the source
\usepackage{animate}
\usepackage{tikz}
\usepackage{pgf-pie}
\usepackage{fontawesome5}
\usepackage{ifthen}
\ifthenelse{\isundefined{\everyframe}}{%
	% If we're compiling this file via \input, then these variables are already defined.
	% In the other case, we need to define them
	\newcommand{\everyframe}{5}
	\definecolor{ubRed}{HTML}{E6002E}
	% complementary colors from https://www.sessions.edu/color-calculator/
	\definecolor{ubRedComplementary}{HTML}{2EE600}
	}{}
\begin{document}
\begin{animateinline}[every=\everyframe]{25}
	\multiframe{180}{n=1+2}{%
		\begin{tikzpicture}
			\pgfdeclarelayer{background}
			\pgfsetlayers{background,main}
			%Help lines
			%\draw[<->] (-2.25,0) -- (2.25,0);
			%\draw[<->] (0,-2.25) -- (0,2.25);
			%\draw[help lines,step=1cm,ultra thin] (-2.45,-2.45) grid (2.45,2.45);%
			% Stuff that stays put
			% X-ray source
			\fill[ubRed] (-0.25,1) rectangle node (source) {} +(0.5,0.5);
			\draw[fill=yellow] (0,1.75) circle (0.1545);%
			\pie[draw=yellow,color={black,yellow},sum=360,hide number,radius=0.1236,pos={0,1.75}]{60/,60/,60/,60/,60/,60/}%
			\draw[color=yellow,fill=black] (0,1.75) circle (0.0309);%
			% Detector
			\fill[gray] (-0.5,-1.75) rectangle node (detector) {} +(1,0.25);
			% Detector edges to point the x-ray cone edges at
			\coordinate (dl) at (-0.45,-1.75);
			\coordinate (dr) at (0.45,-1.75);
			% X-ray cone
			\begin{pgfonlayer}{background}
				\fill[gray,semitransparent] (source.center) -- (dl) -- (dr) -- cycle;
			\end{pgfonlayer}
			% Stuff that moves
			\mode<beamer>{%
				\begin{scope}[rotate around={\n:(0,0.5)}]
				}
				% Rotation arc
				\draw[->, line cap=rect] (0.618,0.5) arc [start angle=0, end angle=180, radius=0.618];
				\draw[->, line cap=rect] (-0.618,0.5) arc [start angle=-180, end angle=0, radius=0.618];
				% Sample
				\node[ubRedComplementary] at (0,0.5) (sample) {\rotatebox{\n}{\tiny\faFish}};
			\mode<beamer>{%
				\end{scope}
				}
		\end{tikzpicture}
	}
\end{animateinline}
\end{document}

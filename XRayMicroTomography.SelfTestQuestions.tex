\documentclass[a4paper]{exam}
%\documentclass[paper=a4,twoside=true,DIV=calc]{scrartcl}
\usepackage[utf8]{inputenc}
\usepackage[T1]{fontenc}
\usepackage{textcomp}
\usepackage[scaled=.92]{helvet}
\usepackage[english]{babel}
\usepackage{microtype}
\usepackage[obeyall=true,detect-weight=true, binary-units=true]{siunitx}
\usepackage{gitinfo2}
\usepackage{graphicx}
\usepackage[margin=15pt]{subcaption}
\usepackage{tikz}
\usepackage{shadowtext}	% for shadowed text on the scalebar
	\shadowoffset{1pt}	% ideally the same as on line 13...
	\shadowcolor{black}	% ideally the same as on line 13...
\usepackage{hyperref}

\newcommand{\imsize}{\linewidth}% default width of image
\newlength\imagewidth% needed for scalebar
\newlength\imagescale% needed for scalebar
\tikzset{shadowed/.style={preaction={transform canvas={shift={(1pt,-1pt)}},draw=black, thick}}}% needed for scalebar

\title{Self-test questions for the \href{https://ilias.unibe.ch/goto_ilias3_unibe_sess_1555744.html}{Lecture on X-ray microtomography}}
\author{David Haberthür}
\date{\today}

% Helvetica all over 
\renewcommand{\familydefault}{\sfdefault}

\begin{document}

\maketitle

\begin{questions}

\question Fully describe the path of the X-rays once they are emitted from the source until they are converted to an image.
	Can you describe all interactions with the sample and what needs to happen that we get a projection image on disk?

\question How does one change the voxel size in a desktop micro-CT scanner?

\question Why can't you see the blood vessels in \autoref{fig:1265}, but see them in \autoref{fig:5158}?
	Hint: It's not because they have been scanned on different machines or at a slightly different voxel sizes\ldots

\begin{figure}[h]
	\centering
	\begin{subfigure}[t]{0.495\textwidth}
		\pgfmathsetlength{\imagewidth}{\imsize}%
		\pgfmathsetlength{\imagescale}{\imagewidth/3295}%
		\def\x{2036}% scalebar-x starting at golden ratio of image width of 3295px = 2036
		\def\y{1343}% scalebar-y at 90% of image height of 1492px = 1343
		\begin{tikzpicture}[x=\imagescale,y=-\imagescale]
			\node[anchor=north west, inner sep=0pt, outer sep=0pt] at (0,0) {\includegraphics[width=\imagewidth]{./images/Mouse1265_resliced_MIP500um}};
			% 3295.000px = 26.2282mm -> 100px = 796.000um -> 62.814px = 500um, 12.563px = 100um
			%\draw[|-|,blue,thick] (0,746) -- (3295,746) node [sloped,midway,above,fill=white,semitransparent,text opacity=1] {\SI{26.2282}{\milli\meter} (3295px) TEMPORARY!};
			\draw[|-|,white,thick,shadowed] (\x,\y) -- (\x+628.14,\y) node [midway,above] {\shadowtext{\SI{5}{\milli\meter}}};
		\end{tikzpicture}%
		\caption{\emph{Mouse 1265}, which you might remember from the visualization on slide 4.
			Scanned on a SkyScan1172 at an isotropic voxel size of \SI{7.96}{\micro\meter}.
			No blood vessels are visible, neither in the brain nor in the mandibles.}
		\label{fig:1265}
	\end{subfigure}%
	\hfill%
	\begin{subfigure}[t]{0.495\textwidth}
		\pgfmathsetlength{\imagewidth}{\imsize}%
		\pgfmathsetlength{\imagescale}{\imagewidth/2358}%
		\def\x{1457}% scalebar-x starting at golden ratio of image width of 2358px = 1457
		\def\y{1415}% scalebar-y at 90% of image height of 1572px = 1415
		\begin{tikzpicture}[x=\imagescale,y=-\imagescale]
			\node[anchor=north west, inner sep=0pt, outer sep=0pt] at (0,0) {\includegraphics[width=\imagewidth]{./images/Mouse5158_resliced_MIP500um}};
			% 2358.000px = 24.759mm -> 100px = 1050.000um -> 47.619px = 500um, 9.524px = 100um
			%\draw[|-|,blue,thick] (0,786) -- (2358,786) node [sloped,midway,above,fill=white,semitransparent,text opacity=1] {\SI{24.759}{\milli\meter} (2358px) TEMPORARY!};
			\draw[|-|,white,thick,shadowed] (\x,\y) -- (\x+476.19,\y) node [midway,above] {\shadowtext{\SI{5}{\milli\meter}}};
		\end{tikzpicture}%
		\caption{Mouse 5158.
		Scanned on a SkyScan1272 at an isotropic voxel size of \SI{10.5}{\micro\meter}.
		The blood vessels are nicely visible.
		The bright, elongated structure at the top of the skull is related to the experiment that was performed with these mice.}
		\label{fig:5158}
	\end{subfigure}	
	\caption{\href{https://en.wikipedia.org/wiki/Maximum_intensity_projection}{MIP}s from a region of approximately \SI{500}{\micro\meter} around the sagittal centerline of two different mouse heads.}
\end{figure}

\end{questions}


\end{document}

\documentclass[a4paper]{exam}

\usepackage[utf8]{inputenc}
\usepackage[T1]{fontenc}
\usepackage{textcomp}
\usepackage[scaled=.92]{helvet}
\usepackage[english]{babel}
\usepackage{microtype}
\usepackage[detect-all=true]{siunitx}
\usepackage{gitinfo2}
\usepackage{graphicx}
\usepackage[margin=10pt]{subcaption}
\usepackage{tikz}
\tikzset{shadowed/.style={preaction={transform canvas={shift={(1pt,-1pt)}},draw=black}}}
\usepackage{shadowtext}
\shadowoffset{1pt}
\shadowcolor{black}
\usepackage{xspace}
\usepackage{hyperref}

\newcommand{\imsize}{\linewidth}% default width of image
\newlength\imagewidth% needed for scalebar
\newlength\imagescale% needed for scalebar
\newcommand{\uct}{{\textmu}CT\xspace}% make our life easier
\newcommand{\mct}{micro-CT\xspace}% make our life easier

\title{Self-test questions for the lecture on \href{https://ilias.unibe.ch/goto_ilias3_unibe_sess_2774509.html}{X-ray microtomography}}
\author{David Haberthür}
\date{\today\xspace| v. \gitAbbrevHash}

% Helvetica all over
\renewcommand{\familydefault}{\sfdefault}

\begin{document}

\maketitle

\begin{questions}

  \question{}
  Describe what happens with the the X-rays once they are emitted from the X-ray source until they are converted to an image in the detector of a desktop \mct{} or large human CT.\@
  Can you describe all interactions with the sample and what needs to happen until we get a projection image on disk?

  \question{}
  In a desktop \mct{} the researcher can change the voxel size of the image acquisition nearly continuously (in a certain bandwidth).
  This is not possible in a large CT which is used for imaging humans in a hospital.
  Why is this so?

  \question{}
  A researcher wants to image a sample with approximately \qty{30}{\centi\meter} with a very small voxel size (let's say below \qty{1}{\micro\meter}).
  Is this possible or not?
  Explain your reasoning.

  \question{}
  Why can you not detect any blood vessels in \autoref{fig:1265}, while they are nicely visualized  in \autoref{fig:5158}?
  Different visualizations of \emph{Mouse 1265} in \autoref{fig:1265} have been shown throughout the lecture.
  The contrast in the skull looks quite equal in both images.
  Both samples have been scanned on different machines, with slighly different settings, resulting in a different voxel sizes.
  The different voxel size is not the cause, though.

  \begin{figure}[h]
    \centering%
    \begin{subfigure}[t]{0.5\linewidth}%
      \pgfmathsetlength{\imagewidth}{\imsize}%
      \pgfmathsetlength{\imagescale}{\imagewidth/3295}%
      \def\x{2036}% scalebar-x starting at golden ratio of image width of 3295px = 2036
      \def\y{1343}% scalebar-y at 90% of image height of 1492px = 1343
      \begin{tikzpicture}[x=\imagescale,y=-\imagescale]
        \node[anchor=north west, inner sep=0pt, outer sep=0pt] at (0,0) {\includegraphics[width=\imagewidth]{./images/Mouse1265_resliced_MIP500um}};
        % 3295.000px = 26.2282mm -> 100px = 796.000um -> 62.814px = 500um, 12.563px = 100um
        %\draw[|-|,blue,thick] (0,746) -- (3295,746) node [sloped,midway,above,fill=white,semitransparent,text opacity=1] {\qty{26.2282}{\milli\meter} (3295px) TEMPORARY!};
        \draw[|-|,white,thick,shadowed] (\x,\y) -- (\x+628.14,\y) node [midway,above] {\shadowtext{\qty{5}{\milli\meter}}};
      \end{tikzpicture}%
      \caption{\emph{Mouse 1265}.
        Scanned on a SkyScan1172 at an isotropic voxel size of \qty{7.96}{\micro\meter}.
      No blood vessels are visible, neither in the brain nor in the mandibles.}%
      \label{fig:1265}%
    \end{subfigure}%
    \hfill%
    \begin{subfigure}[t]{0.5\linewidth}%
      \pgfmathsetlength{\imagewidth}{\imsize}%
      \pgfmathsetlength{\imagescale}{\imagewidth/2358}%
      \def\x{1457}% scalebar-x starting at golden ratio of image width of 2358px = 1457
      \def\y{1415}% scalebar-y at 90% of image height of 1572px = 1415
      \begin{tikzpicture}[x=\imagescale,y=-\imagescale]
        \node[anchor=north west, inner sep=0pt, outer sep=0pt] at (0,0) {\includegraphics[width=\imagewidth]{./images/Mouse5158_resliced_MIP500um}};
        % 2358.000px = 24.759mm -> 100px = 1050.000um -> 47.619px = 500um, 9.524px = 100um
        %\draw[|-|,blue,thick] (0,786) -- (2358,786) node [sloped,midway,above,fill=white,semitransparent,text opacity=1] {\qty{24.759}{\milli\meter} (2358px) TEMPORARY!};
        \draw[|-|,white,thick,shadowed] (\x,\y) -- (\x+476.19,\y) node [midway,above] {\shadowtext{\qty{5}{\milli\meter}}};
      \end{tikzpicture}%
      \caption{\emph{Mouse 5158}.
        Scanned on a SkyScan1272 at an isotropic voxel size of \qty{10.5}{\micro\meter}.
        The blood vessels are nicely visible.
      The bright, elongated structure at the top of the skull is related to the experiment that was performed with these mice, but also not the cause.}%
      \label{fig:5158}%
    \end{subfigure}
    \caption{\href{https://en.wikipedia.org/wiki/Maximum_intensity_projection}{MIP}s from a region of approximately \qty{500}{\micro\meter} around the sagittal centerline of two different mouse heads, scanned on two different desktop \mct{} scanners.}
  \end{figure}

\end{questions}

\end{document}
